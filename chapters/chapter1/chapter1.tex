\chapter{Introduction}


\section{Motivation}
Ethereum's native currency, Ether or ETH, is the second-largest cryptocurrency in the world in terms of total market capitalization at the time of writing this thesis. Bitcoin is in first place but it only supports a limited scripting language that makes it difficult to deploy smart contracts. Thus, Ethereum is the most widely used blockchain that enables the deployment of verbose smart contracts with wide functionality. 

Smart contracts are pieces of code that run on blockchains such as Ethereum and eliminate the need of use of a trusted third party to run the logic. Also, Ethereum has its own native currency, by which the users and smart contracts can make payments automatically.
Therefore, there is a growing tendency to move from the development of traditional applications which relies on intermediate companies (Web2) to Decentralized Applications (Dapps), Web3, that relies on smart contracts. 
 
Smart contracts usually have control on large amounts of cryptocurrencies in their custody or in the custody of another smart contract. Also blockchains are immutable, which means that the transactions are irreversible because of nature of blockchains . 
We have seen that vulnerabilities in code bases of smart contracts result in significant financial loss in DeFi and Ethereum and just in 2021, 1.3 billions of dollars are lost reported by Certik~\cite{certikReport}.


Technical and economical nuanced differences between Web2 apps and Web3 Dapps is a huge burden in the development of Web3 Dapps. Even professional experienced Web3 developers, face problems regarding these differences. Also, a wave of new developers have changed their careers to develop smart contracts on Ethereum in recent years who have problems to grab the differences between Web2 development and Web3 development. In this dissertation, we try to shed light on some of these nuanced differences to help developers deal with the burdens in Web3 development. Ethereum blockchain may be similar to a cloud service to some developers using it for the first time. But, there are nuances where smart contracts are immutable and cannot be changed. Also smart contracts do not have access to real-world data and need oracle systems for this purpose.


\section{Thesis Statement}
The primary aim of this dissertation is to shed light on specific problems Web3 developers facing in Dapp development such as dealing with oracle systems, upgrading the systems and dealing with stablecoins. Over the course of time, developers tried to port more and more Web2 functionalities into Web3, but it almost always requires new risks or trusted third parties. Some of these risks and trust points are explored in this dissertation by answering questions below:

\begin{enumerate}
    \item \textbf{Question 1:} Studying ways to upgrade code on the blockchain that is supposed to be immutable in case a new feature or a bug fix is needed?
    \item \textbf{Question 2:} Researching the ways to have access to real-world data on the blockchain and alternatives of using an API to fetch data on a blockchain?
    \item \textbf{Question 3:} What if a Dapp needs payment in a currency like USD and not the native currency of the blockchain which is volatile? What are the solutions and risks regarding the payment solutions in blockchains?
\end{enumerate}


\section{Outline and Contributions}
The rest of this dissertation is organized as follows. In Chapter~\ref{ch:upgrade}, we summarize and evaluate six patterns, developed on Ethereum to enable upgradeability of smart contracts. Modern smart contracts use software tricks to enable upgradeability, raising the research questions of \textit{how} upgradeability is achieved and \textit{who} is authorized to make changes. We develop a measurement framework for finding how many upgradeable contracts are on Ethereum that use certain prominent upgrade patters. We also measure how they implement access control over their upgradeability: about 50\% are are controlled by a single Externally Owned Address (EOA), and about 13\% are controlled by multi-signature wallets in which a limited number of persons can change the whole logic of the contract which is a risk to the Ethereum ecosystem.

In Chapter~\ref{ch:oracle} we describe that one fundamental limitation of blockchain-based smart contracts is that they execute in a closed environment. Thus, they only have access to data and functionality that is already on the blockchain, or is fed into the blockchain. Any interactions with the real world need to be mediated by a bridge service, which is called an oracle. As decentralized applications mature, oracles are playing an increasingly prominent role. With their evolution comes more attacks, necessitating greater attention to their trust model. We also dissect the design alternatives for oracles, showcase attacks, and discuss attack mitigation strategies.


In Chapter~\ref{ch:dai} we shed light on a number of Ethereum projects for stablecoins and synthetic assets use the same core mechanism for fixing the price of an asset. In this chapter, we distil this shared approach into a primitive we call red-black coins. We use a model to demonstrate the primitive's financial characteristics and to reason about how it should be priced. Real world projects do not use the red-black coin primitive in isolation but lay on other mechanisms and features to provide fungibility and to reduce exposure to price drops. One mechanism is called liquidation, however liquidation is hard to analyze as it relies on human behavior and could produce unintended economic consequences. Therefore we additionally develop a design landscape for extending the red-black coin primitives and put forward a research agenda for alternatives to liquidation.

In Chapter ~\ref{chap:conclusion} we provide some concluding remarks and future research directions.



%\paragraph{Academic Paper. }The work on oracles discussed in this dissertation in Chapter~\ref{ch:oracle} has been peer-reviewed and published in the following article:
%
%\begin{displayquote}
%Eskandari, Shayan, et al. "Sok: Oracles from the ground truth to market manipulation." Proceedings of the 3rd ACM Conference on Advances in Financial Technologies. 2021.
%\end{displayquote}
%
%\paragraph{Academic Paper. }The work on stablecoins discussed in this dissertation in Chapter~\ref{ch:dai} has been peer-reviewed and published in the following article:
%
%\begin{displayquote}
%Salehi, Mehdi, Jeremy Clark, and Mohammad Mannan. "Red-Black Coins: Dai without liquidations." International Conference on Financial Cryptography and Data Security. Springer, Berlin, Heidelberg, 2021
%\end{displayquote}