\chapter{Introduction}


\section{Motivation}
Ethereum is the second-largest cryptocurrency in the world in term of total market cap at the time of writing this thesis. Bitcoin is the first cryptocurrency in total market cap and it supports a limited scripting language. But, Ethereum is the most widely used blockchain that enables developing rich codes (smart contracts) not just limited to a scripting language.

Smart contracts are pieces of codes that run on blockchains such as Ethereum and eliminate the need of use of a trusted third party to run the logic. Also Ethereum has its own native currency, Ether, by which the users and smart contracts can make payments automatically.
Therefore, there is a growing tendency to move from development of traditional applications which relies on intermediate companies, Web2, to Decentralized Applications (Dapps), Web3, that relies on smart contracts. 

Smart contracts usually keeps huge amount of cryptocurrencies in their custody. Also blockchains are immutable, which means that the transactions are irreversible because of nature of blockchains . So, a small mistake or a buggy smart contract could result in a huge hack and a disaster.


A wave of developers changing their careers to develop smart contracts on Ethereum in recent years. In this thesis we try to shed light on some technical and economical difference between web2 apps and web3 Dapps that a new developer should know about them for developing new tools and to try give them a clue about ``Nuances of Ethereum Blockchain'' and development on Ethereum and how it is different to developing a web2 application and using cloud services.  


\section{Thesis Statement}
The primary aim of this dissertation is to shed light on a specific differences on developing traditional applications on web2 and the Decentralized Applications on web3. The main research questions explored in this dissertation are:

\begin{enumerate}
    \item \textbf{Question 1:} How to upgrade codes on the blockchain that is supposed to be immutable in case a new feature or a bug fix needed?
    \item \textbf{Question 2:} How to have access to real-world data on the blockchain? Is it possible to use an API to fetch data on blockchain?
    \item \textbf{Question 3:} What if a Dapp needs payment in a currency like USD and not native currency of the blockchain which is volatile? What are the solutions and risks regarding them?
\end{enumerate}


\section{Outline and Contributions}
The rest of this dissertation is organized as follows. In Chapter~\ref{ch:upgrade} we summarizes and evaluates six patterns, developed on Ethereum to enable upgradeability of smart contracts. Modern smart contracts use software tricks to enable upgradeability, raising the research questions of \textit{how} upgradeability is achieved and \textit{who} is authorized to make changes. We develop a measurement framework for finding how many upgradeable contracts are on Ethereum that use certain prominent upgrade patters. We also measure how they implement access control over their upgradeability: about 50\% are are controlled by a single Externally Owned Address (EOA), and about 13\% are controlled by multi-signature wallets in which a limited number of persons can change the whole logic of the contract which is a great risk to the Ethereum ecosystem.

In Chapter~\ref{ch:oracle} we describe that one fundamental limitation of blockchain-based smart contracts is that they execute in a closed environment. Thus, they only have access to data and functionality that is already on the blockchain, or is fed into the blockchain. Any interactions with the real world need to be mediated by a bridge service, which is called an oracle. As decentralized applications mature, oracles are playing an increasingly prominent role. With their evolution comes more attacks, necessitating greater attention to their trust model. We also dissect the design alternatives for oracles, showcase attacks, and discuss attack mitigation strategies.


In Chapter~\ref{ch:dai} we shed light on a number of Ethereum projects for stablecoins and synthetic assets use the same core mechanism for fixing the price of an asset. In this chapter, we distil this shared approach into a primitive we call red-black coins. We use a model to demonstrate the primitive's financial characteristics and to reason about how it should be priced. Real world projects do not use the red-black coin primitive in isolation but lay on other mechanisms and features to provide fungibility and to reduce exposure to price drops. One mechanism is called liquidation, however liquidation is hard to analyze as it relies on human behavior and could produce unintended economic consequences. Therefore we additionally develop a design landscape for extending the red-black coin primitives and put forward a research agenda for alternatives to liquidation.

In Chapter ~\ref{chap:conclusion} we provide some concluding remarks and future research directions.

