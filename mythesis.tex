\documentclass[letterpaper,12pt,onecolumn,final]{report}

\pdftrailerid{}
\pdfsuppressptexinfo15
\pdfminorversion=4

%% MANDATORY PACKAGES
\usepackage{cuthesis}         % Concordia's thesis style
\usepackage[english]{babel}   % load english localization
\usepackage{type1ec}          % type 1 font
\usepackage[T1]{fontenc}      % correct some font representation, needs cm-super fonts
\usepackage{times}            % use Times New Roman font
\usepackage[titletoc,title]{appendix}     % include Appendix command, add to ToC
\usepackage{setspace}         % control double/single line spacing
\usepackage[pdftex]{graphicx}
\usepackage{caption,tabularx,booktabs}
\usepackage{array}
\usepackage{csquotes}
\newcolumntype{C}[1]{>{\centering\arraybackslash}p{#1}}
%% OPTIONAL PACKAGES
%\counterwithout{footnote}{chapter}        % do no reset footnote # between chapters
\usepackage[hyphens]{url}     % print links
\usepackage{hyperref}         % provides hyperlinks (text different than link)
%\usepackage[hyphenbreaks]{breakurl}       % break long URL after hyphens
\hypersetup{
	colorlinks=true,
	breaklinks=true,
	linkcolor=black,
	citecolor=black,
	urlcolor=black,
	filecolor=black,
	linktoc=all,
}
\usepackage{graphicx}
%\graphicspath{{img/}}

\usepackage{blindtext}

%% CUSTOM MACROS
\usepackage{xspace}
\newcommand{\etal}{\textit{et al.}\xspace}
\newcommand{\etc}{\textit{etc.}\xspace}
\newcommand{\ie}{\textit{i.e.,}\xspace}
\newcommand{\eg}{\textit{e.g.,}\xspace}
\newcommand{\cf}{\textit{cf.}\xspace}
\newcommand{\supra}{\textit{Supra}\xspace}
\newcommand{\nee}{\textit{n\'ee}\xspace}
\newcommand{\aka}{\textit{a.k.a.,}\xspace}

% = = = Arrow -> (\lt)
\newcommand{\lt}{$\rightarrow$\xspace}

% = = = Keywords (kw)
\newcommand{\kw}[1]{\textsf{#1}}

% = = = Colored text (textblue)
\newcommand{\textblue}[1]{\textcolor{blue}{#1}}

% = = = Compact Lists (compactlist, compactlistn)
\newenvironment{compactlist}
  {\begin{itemize} 
  \setlength{\itemsep}{0pt} 
  \setlength{\parskip}{0pt}} 
  {\end{itemize}}
  
\newenvironment{compactlistn}
  {\begin{enumerate} 
  \setlength{\itemsep}{0pt} 
  \setlength{\parskip}{0pt}} 
  {\end{enumerate}}
  
\renewcommand{\labelitemi}{$\bullet$}
  

%------------------------Crypto----------------------%  

% = = = Zp, Gq and Zq
\newcommand{\Zp}{\mathbb{Z}^{*}_{p}}
\newcommand{\Zq}{\mathbb{Z}_{q}}
\newcommand{\Gq}{\mathbb{G}_{q}}

% = = = Encryption, etc.
\newcommand{\Enc}[1]{\mathsf{Enc}(#1)}
\newcommand{\EncB}[1]{\llbracket #1 \rrbracket}
\newcommand{\ReRand}[1]{\mathsf{ReRand}(#1)}
\newcommand{\Hash}[1]{\mathcal{H}(#1)}
\newcommand{\Sign}[1]{\mathsf{Sig}(#1)}
\newcommand{\Comm}[1]{\mathsf{Comm}(#1)}
\newcommand{\Open}[1]{\mathsf{Open}(#1)}

% = = = Tuples
\newcommand{\tuple}[1]{\left \langle #1 \right \rangle}


%-------------------Custom for Paper----------------------%

% = = = Name
\newcommand{\Name}{\textsf{System Name}\xspace}
\newcommand{\dai}{\textsf{Dai}\xspace}
\newcommand{\cdp}{\textsf{CDP}\xspace}
\newcommand{\vault}{\textsf{Vault}\xspace}

%\newcommand{\tickyes}{{\small\checkmark}}
%\newcommand{\tickno}{{\small$\times$}}
% TABLES:
\usepackage{adjustbox}
\newcommand{\headrow}[1]{\multicolumn{1}{c}{\adjustbox{angle=30,lap=\width-0.5em}{#1}}}
\newcommand{\full}{$\bullet$}
\newcommand{\prt}{$\circ$}
\newcommand{\none}{$\times$}
%% CUSTOM COMMANDS
%\newcommand{\subhead}[1]{\noindent{\textbf{#1.}}}
\input{setup/packages}
%% THESIS SETTINGS
\author{Mehdi Salehi}
\title{An Analysis of Upgradeability, Oracles, and Stablecoins in the Ethereum Blockchain}

% As of 2019, title is no longer used...
%\titleOfPhDAuthor{Mr.}         % or Ms., Mrs., Miss, etc. (only for PhD's)

% if PhD, uncomment:
%\PhD
% else if Master's, uncomment:
\mastersDegree{Master of Applied Science}

\program{Information and Systems Engineering}
%\program{Computer Science}
\dept{The Concordia Institute\\for\\Information Systems Engineering}
%\dept{The Department\\of\\Computer Science and Software Engineering}

%% See current GPD at https://www.concordia.ca/admissions/graduate/programs/contacts.html
\GpdOrChairOfDept{}
\isGpd % Chair by default
%% See current Dean at  https://www.concordia.ca/ginacody/about/leadership/office-dean/dean-of-engineering-and-computer-science.html
\deanOfENCS{Dr.\ Mourad Debbabi} 
\chairOfCommittee{Dr.\ Amr Youssef}
\examinerFirst{Dr.\ Amr Youssef}
\examinerSecond{Dr.\ Ivan Pustogarov}
\supervisor{Dr. Jeremy Clark}
%% Following two lines are required if you have a co-supervisor
\hasCosupervisor
\coSupervisor{Dr. Mohammad Mannan}

%% Comment to use current month, needs to match initial submission
\submitmonth{April}
\submityear{2022}
%% Comment if date of defence is unknown yet, fill for final submission
\defencedate{April 1, 2022}


%%%%%%%%%%%%%%%%%%%%%%%%%%%%%%%%%%%%%%%%%%%%%%%%%%%%%%%%%%%%%%%%%%%%%%%%%%%%%%%

\doublespacing
\begin{document}

\begin{abstract}
{%trick to force double spacing in the abstract, otherwise some paragraphs may show single spaced
\setstretch{1.6667}

The Ethereum blockchain is a widely adopted global alternative to cloud computing platforms, currently used primarily for financial services. Given the large number of funds held by smart contracts and decentralized applications on top of Ethereum, there are profound security implications for both users and enterprise developers.

Over time, developers have brought more complex logic to Ethereum. For example, contracts often require access to valid, real-world data. In most cases, the system's functionality and security are strongly dependent on the correctness and safeness of the data pushed to the blockchain. One topic of this thesis are oracles---infrastructure added to the blockchain to respond to this need.
As contract code becomes more complex, it is increasingly likely that the code has bugs or vulnerabilities. Given smart contracts are immutable and tamper-proof, it seems impossible to upgrade a contract should a fix or patch be needed. Another topic of this thesis examines contract deployment patterns that enable and handle the upgradeability of smart contracts in Ethereum. Finally, the thesis also considers an application of oracle technology: payments made in stable currencies such as USD and not blockchain native currencies such as ETH, which are volatile in price. This thesis explains each topic in detail, evaluating the security risks of each, and examining any consequences for user trust and the degree of decentralization. 

% JC: try to keep to 1 page

%  broad adoption of decentralized applications brought another need in the service layers. The payments by user need to be This brings the idea of developing stablecoins pegged to a stable currency like the USD.

%Adding upgradeability patterns, oracles, and stablecoins to decentralized applications and smart contract logic adds significant security risks and the need to use a trusted third party, which is against disintermediation in the blockchain ecosystem. 
%  central trusted points that will be added because of each component. We also try to extend the design choices for each component to give the designer a broader spectrum, which may help make the systems more secure.

}
\end{abstract}

%\doublespacing
\begin{acknowledgments}
  I would like to thank my supervisors, Dr. Mohammad Mannan and Dr. Jeremy Clark, for their continued support and guidance during the course of my master's degree. Their constant support gave life to this project and made this thesis possible. I have learned a lot from them and would like to express my gratitude for their patience, motivation, and immense knowledge. I am fortunate to work under the close guidance of them, who inspired me with bright ideas, helpful comments, suggestions, and insights that have contributed to the improvement of this work.

  I would like to thank Shayan Eskandari, my colleague, for mentoring me during my master's and sharing his experience, and being by my side through this journey. I want to thank Mahsa Mousavi, my lab-mate, for her continuous feedback and support during our discussions and meetings. She has been very generous with her help during the course of my research.
  
  A very special word of thanks to my best friend in my life Nikoo Farvardin, who has always been a major source of support when things would get a bit discouraging. Her presence was significant in a process that is often felt as tremendously solitaire. She gave me support and help, discussed ideas, and prevented several wrong turns.
  
  In the end, I would like to acknowledge the unconditional affection and continuous support of my parents, my sister Neda, and my brother Saleh. They always fulfill all my silly demands and keep faith in my ability. They have always been the best inspiration in my life. I would like to dedicate this thesis to them. This journey would not have been possible without their encouragement and support. I am incredibly lucky to have them in my life.
\end{acknowledgments}


%%%%%%%%%%%%%%%%%%%%%%%%


%%%%%%%%%%%%%%%%%%%%%%%%


%%%%%%%%%%%%%%%%%%%%%%%%
\chapter{Introduction}


\section{Motivation}
Ethereum is the second-largest cryptocurrency in the world in term of total market cap at the time of writing this thesis. Bitcoin is the first cryptocurrency in total market cap and it supports a limited scripting language. But, Ethereum is the most widely used blockchain that enables developing rich codes (smart contracts) not just limited to a scripting language.

Smart contracts are pieces of codes that run on blockchains such as Ethereum and eliminate the need of use of a trusted third party to run the logic. Also Ethereum has its own native currency, Ether, by which the users and smart contracts can make payments automatically.
Therefore, there is a growing tendency to move from development of traditional applications which relies on intermediate companies, Web2, to Decentralized Applications (Dapps), Web3, that relies on smart contracts. 

Smart contracts usually keeps huge amount of cryptocurrencies in their custody. Also blockchains are immutable, which means that the transactions are irreversible because of nature of blockchains . So, a small mistake or a buggy smart contract could result in a huge hack and a disaster.


A wave of developers changing their careers to develop smart contracts on Ethereum in recent years. In this thesis we try to shed light on some technical and economical difference between web2 apps and web3 Dapps that a new developer should know about them for developing new tools and to try give them a clue about ``Nuances of Ethereum Blockchain'' and development on Ethereum and how it is different to developing a web2 application and using cloud services.  


\section{Thesis Statement}
The primary aim of this dissertation is to shed light on a specific differences on developing traditional applications on web2 and the Decentralized Applications on web3. The main research questions explored in this dissertation are:

\begin{enumerate}
    \item \textbf{Question 1:} How to upgrade codes on the blockchain that is supposed to be immutable in case a new feature or a bug fix needed?
    \item \textbf{Question 2:} How to have access to real-world data on the blockchain? Is it possible to use an API to fetch data on blockchain?
    \item \textbf{Question 3:} What if a Dapp needs payment in a currency like USD and not native currency of the blockchain which is volatile? What are the solutions and risks regarding them?
\end{enumerate}


\section{Outline and Contributions}
The rest of this dissertation is organized as follows. In Chapter~\ref{ch:upgrade} we summarizes and evaluates six patterns, developed on Ethereum to enable upgradeability of smart contracts. Modern smart contracts use software tricks to enable upgradeability, raising the research questions of \textit{how} upgradeability is achieved and \textit{who} is authorized to make changes. We develop a measurement framework for finding how many upgradeable contracts are on Ethereum that use certain prominent upgrade patters. We also measure how they implement access control over their upgradeability: about 50\% are are controlled by a single Externally Owned Address (EOA), and about 13\% are controlled by multi-signature wallets in which a limited number of persons can change the whole logic of the contract which is a great risk to the Ethereum ecosystem.

In Chapter~\ref{ch:oracle} we describe that one fundamental limitation of blockchain-based smart contracts is that they execute in a closed environment. Thus, they only have access to data and functionality that is already on the blockchain, or is fed into the blockchain. Any interactions with the real world need to be mediated by a bridge service, which is called an oracle. As decentralized applications mature, oracles are playing an increasingly prominent role. With their evolution comes more attacks, necessitating greater attention to their trust model. We also dissect the design alternatives for oracles, showcase attacks, and discuss attack mitigation strategies.


In Chapter~\ref{ch:dai} we shed light on a number of Ethereum projects for stablecoins and synthetic assets use the same core mechanism for fixing the price of an asset. In this chapter, we distil this shared approach into a primitive we call red-black coins. We use a model to demonstrate the primitive's financial characteristics and to reason about how it should be priced. Real world projects do not use the red-black coin primitive in isolation but lay on other mechanisms and features to provide fungibility and to reduce exposure to price drops. One mechanism is called liquidation, however liquidation is hard to analyze as it relies on human behavior and could produce unintended economic consequences. Therefore we additionally develop a design landscape for extending the red-black coin primitives and put forward a research agenda for alternatives to liquidation.

In Chapter ~\ref{chap:conclusion} we provide some concluding remarks and future research directions.


\chapter{Background}
\label{chap:background}
This chapter covers background information about Ethereum basics, details about Ethereum Virtual Machine (EVM), explanations on contract creation in Ethereum, and the way users interact with Ethereum network. Also, there is a section that briefly explains some use-cases and Dapps on Ethereum blockchain.
 
\section{Ethereum Background}

Ethereum is a state machine that is globally accessible and its Ethereum Virtual Machine, applies the changes to the state based on the rules defined for the Ethereum network. Ethereum is often described as ``world computer'' because anybody from anywhere can have access to Ethereum and use the stack of Ethereum and the EVM to execute their computations. 

Ethereum is the next generation of blockchains compared to Bitcoin. Bitcoin is just handling payments and some specific limited logic based on a limited scripting language which is called ``Bitcoin Scripting Language''. For instance, Bitcoin scripts cannot handle loops. But in Ethereum, EVM can handle more general and complex code named \textit{Smart Contracts}. EVM is quasi-Turing-complete state machine. It is ``quasi'' because computation in Ethereum is bounded to specific number of gas, which is a unit of measurement of computations and storage resources in Ethereum. 

Users can work with Ethereum blockchain by sending transactions to the network. They can just transfer Ether (or ETH), the native currency of Ethereum blockchain, or they can do their desired computations by executing smart contract functionalities in Ethereum. Users must pay the fee to work with Ethereum. This fee is calculated in gas, and it is based on the amount of computations and storage load on the Ethereum network and also network congestion.  

For more details about the basics of Ethereum, we refer Antonopoulos and Wood~\cite{antonopoulos2018mastering} (chapter one and two) to the readers who are eager to learn more about Ethereum.
In the further parts of this section, we explain the more detailed aspects of Ethereum Virtual Machine (EVM), Ethereum infrastructure, and Decentralized Applications (Dapps) running on top of Ethereum. 

Smart contracts (in Ethereum) are usually written in high-level programming languages such as Solidity and Vyper and compiled into a low-level EVM bytecodes using a compiler (\eg Solc). An EVM bytecode is a binary string that is interpretable by EVM. Then EVM bytecode will be deployed to the Ethereum blockchain.

To understand the process of deployment of a smart contract one should have familiarity with different types of accounts in Ethereum. There are two types of Ethereum accounts: \textit{Externally Owned Account (EOA)} and \textit{Contract Account}. 
EOA is an account that is controlled by a private key and keeps balance of Ethers that the account has and also a \textit{nonce}, which indicates the number of transactions sent by the account and acts like a counter to mitigate replay attacks. Any external actor can generate a random private key and use ECDSA encryption algorithm. Ethereum uses SECP-256k1 curve for ECDSA as described in the Ethereum yellow paper~\cite{wood2014ethereum}.
The EOA addresses can create new transactions, sign them with the private key related to the account and sent them to Ethereum network to be confirmed and recorded on the Ethereum blockchain. This transaction can:
\begin{enumerate}
    \item Send Ether to another EOA.
    \item Create a new \textit{Contract Account} by sending a bytecode.
    \item Call a \textit{Contract Account} to execute a logic.
\end{enumerate}

Each transaction type is explained below:

\paragraph{Sending Ether.} 
For sending Ether into another EOA account, the sender should specify the address which should be an EOA and not a contract account and also an amount to be sent.

\paragraph{Creating a new Contract Account.}
Creating a new \textit{Contract Account} is a bit tricky. The EOA should send a transaction containing the bytecode of the contract and setting the destination address to \emph{null}. Then EVM will create an address for the newly generated contract. The newly generated account (which is a contract account) keeps four fields: A \textit{nonce} which is similar to the nonce field in EOA accounts which keeps track of number of transactions created by that account; Ether balance which indicates number of Ether the account holds; \textit{Contract Code} which keeps the bytecode of the account which is executed each time a transaction calls the account; \textit{Contract Storage} which is a Merkle Patricia Tree data structure that maintains the data related to the contract account. Detailed explanation of the contract storage can be found in Section~\ref{sec:storagelayout}. 
The other important aspect of contract creation in Ethereum is how EVM generates the address for newly generated contract account. In EOA contract creation, EVM calculates the contract address by using the transaction sender's address and its nonce. The formula is show below:
\begin{equation}
    \label{eqn:create}
    Address = keccak256(rlp.encode(Sender Address, nonce))[12:]
\end{equation}
In this formula \textit{keccak256} is the \textit{sha3} hash used in Ethereum blockchain. Also \textit{RLP} (Recursive Length Prefix) encoding is used in Ethereum to encode arbitrary nested arrays of binary data, such as transactions data. Also at the end, EVM picks the last 20 bytes of the hash by removing the first twelve bytes of the hash.

In addition, contract accounts also can create a new contract account using two different EVM opcodes: \texttt{CREATE} and \texttt{CREATE2}. For transactions originated by a contract account and consist of these two opcodes, EVM will deterministically calculate the address for the new account and then create a contract account with the related four fields explained above. The whole process of contract creation is similar for these two opcodes, but the main difference is how EVM calculates the address for the contract account. For \texttt{CREATE} opcode the calculation formula is the same as Formula~\ref{eqn:create}, which EVM uses for EOA contract creation. But, for \texttt{CREATE2} opcode, EVM uses completely different formula to calculate it. In this type, EVM calculates the address by using sender's address and a bytecode that is sent to be deployed (a.k.a init code). Because the formula does not contain nonce and all other variables are predictable, the newly generated address in this type is \emph{Predictable}. The formula for contract address calculations using \texttt{CREATE2} is:

\begin{equation}
    \label{eqn:create2}
    Address = keccak256(0xff + senderAddress + salt + keccak256(init-code))[12:]    
\end{equation}

Based on the Formula~\ref{eqn:create2}, if the same sender address tries to deploy the same initialization code, the address will be the same. But, EVM does not permit to re-deploy a contract if there is still a contract account related to that address. But, there is a way to delete the previous contract account and then re-deploy the contract on the same address. If the contract contains a \texttt{SELFDESTRUCT} opcode, by calling it, the whole contract account along with its four field will be wiped out and so there is a chance to re-deploy the contract. The whole process is explained in detail in Chapter~\ref{ch:upgrade}. 

\paragraph{Calling a \textit{Contract Account} to execute a logic.}
The last way of interaction of an EOA with Ethereum is sending a transaction by which calling a contract code to execute its logic. If the recipient (destination) of the transaction is a contract account, EVM will start executing its contract code automatically. To explain it better we split the process into two steps describe below:

\paragraph{Transaction creation by user. }Suppose contract A has function ``setNumber'' to set a \textit{uint}\footnote{Unsigned Integer} variable ``num'' that takes the new amount as input. If Alice decides to assign a new number, for instance 3, to the num variable, she should send a transaction specify the recipient as contract A's address. Also transactions to a contract have another field named ``data'' in which the user indicates which function of the contract is going to be called and what are the inputs for the function call. The data part of the transaction is also known as ``Call-Data'' field. Alice should declare that she is going to call ``setNumber'' function with input 3. Defining the target function in EVM is not by putting the name of the function. 

EVM uses \textit{Function Selector} to recognize the functions which is the first four bytes of the Keccak-256 hash of the \textit{function signature}. The function signature is defined as the function name with the list of parameter types divided by a comma - no spaces are used. For instance, for our case the function signature is ``setNumber(uint)''. So, Alice calculates the function signature using this formula:

\begin{equation}
    Function Selector = Bytes4(keccak256(``setNumber(uint)'')) = 0x234a4ac2
\end{equation}

Now Alice has the function selector and can generate the call data of the transaction by putting together the function selector and the encoded inputs in the defined order. The transaction is created and should be signed by Alice and sends to the Ethereum network.

\paragraph{Execution Process. } 
The next step is to describe how EVM interpret and execute the transaction sent by Alice. EVM checks the recipient of the transaction and because it is the address of contract A, it will start executing the bytecode of contract A. EVM first extracts the first four bytes of call data of the transaction in case the transaction calls a contract. Then checks that if the function selectors in the bytecode matches with the first four bytes of the call data. If a function selector is found, it means that the user called that specific function of the contract (in our case, setNumber function of contract A). If EVM cannot find any function selector that matches the first four bytes of call data, then a function named \textbf{Fallback} function will be called. The Fallback function is a function defined in the high level languages (\eg Solidity) and supposed to be executed if EVM could not find function selector equal to the first four bytes of calldata or also if the calldata is empty (\ie sending Ether to the contract without function call). 

In each function, in the bytecode, first EVM will extract the inputs related to the function from calldata and then execute the logic of that function with the provided inputs.

Also it should be mentioned that a contract can also call a function of another contract providing the related function selector and inputs and the other processes are the same. This type of call are called \textbf{Internal Calls}. In literature, it is also called \textbf{Messages} instead of transactions and will be discussed in detail in Section~\ref{sec:txVsMsg}.

There are other technical explanations needed about Ethereum that a reader should know for this dissertation which we will describe below:

\subsection{Run-time Bytecode vs. Contract Creation Code}
%run time and contract creation
In Ethereum the code that is sent to create contract (a.k.a \textbf{Contract Creation Code}), may be different with the code that is stored in the blockchain for that contract account (a.k.a \textbf{Run-time bytecode}). Run-time bytecode is the bytecode which is saved in the blockchain as the code of that specific contract account and each time the contract is called, this code will be executed by EVM. But, Contract Creation code is the input data of the transaction by which a user or a contract tries to deploy a new contract. The Contract Creation code has a field, named \textit{init code}, which is responsible to: 1) make changes to the state of the contract (initializing the storage variables using a constructor), 2) put the run-time bytecode to the memory 3) put the length of the run-time bytecode in the memory 4) put the offset in the memory where the run-time bytecode is saved in the stack 5) execute the \texttt{Return} opcode to push EVM to deploy the contract.\footnote{\url{https://leftasexercise.com/2021/09/05/a-deep-dive-into-solidity-contract-creation-and-the-init-code/}} 

So, the run-time bytecode could be completely different from the contract creation code. It will be discussed in the detail in Chapter~\ref{ch:upgrade} in Section~\ref{sec:metamorphic}.

\subsection{Storage Layout in EVM} \label{sec:storagelayout}
In Ethereum each contract account holds state in its own permanent storage. EVM uses an uncommon storage structure to store the storage state based on the variable types. It uses 32-bytes to 32-bytes key-value mapping to store the data which are zero initialized. Except dynamic arrays and mappings, the other variable types are stored in this structure contiguously one after another starting from slot 0. 

Mappings and dynamic arrays cannot be saved contiguously, because their size is unpredictable. So for dynamic arrays if the storage location after applying the rules ends up at slot P, the size of the array will be saved in this slot and the elements of the array will be saved contiguously starting from slot \textit{keccak256(p)}.

For mapping instead of dynamic array, zero will be saved in slot P. Also element of the mapping with key equal to K is stored in slot \textit{keccak256(h(K), p)} where h is keccak256 of the key value padded to 32 bytes~\cite{wood2014ethereum}.

The storage variables are accessible on-chain if there is a getter function inside the contract that gives the variable amount. Marking a variable as public is also give the opportunity to read the data on-chain because Solidity compiler creates a getter function for all public variables. 

To read the data off-chain, one needs to have access to an Ethereum node which is discussed in Section~\ref{sec:nodes}.


\subsection{Transaction vs. Message}\label{sec:txVsMsg}
We discussed Ethereum transactions in detail in the previous sections. A message is very similar to a transaction but it is produced by another contract instead of an EOA. Messages are the way that contracts calling each others' functions. A message is produces when a contract uses a \texttt{CALL}, \texttt{DELEGATECALL} or \texttt{STATICCALL} opcodes. So, a function from another contract will be executed. The main difference between \texttt{CALL} and \texttt{DELEGATECALL} opcodes is that by \texttt{CALL}ing another contract the storage layout of the the destination contract will be changed. But, \texttt{DELEGATECALL}ing another contract, keeps the contract context. It means that the storage layout of the caller contract will be changed instead of the destintion contract. It is analogues to copy pasting that specific function in the caller contract and running its logic inside the caller contract. There are risks regarding \texttt{DELEGATECALL} opcode and the fact that it keeps context such as \textit{function selector clashes} and \textit{storage layout clashes} which will be discussed in more details in Chapter~\ref{ch:upgrade}, Section~\ref{sec:delegatecall}.

Also it worths mentioning that \texttt{STATICCALL} act completely similar to \texttt{CALL} opcode, except it will be reverted if the message tries to change the state during the call. 
\footnote{\url{https://ethdocs.org/en/latest/contracts-and-transactions/account-types-gas-and-transactions.html}}


%compiler ??!!

\subsection{Off-chain Access to the Blockchain Data}\label{sec:nodes}

To have access to the data stored in Ethereum blockchain, one should have access to an ``Ethereum Node''. Ethereum node refer to running a piece of software called Ethereum client, which implemented the rules and specifications defined by Ethereum yellow paper~\cite{wood2014ethereum} to join and sync with Ethereum network and to keep Ethereum network and data secure and safe.
\footnote{\url{https://ethereum.org/en/developers/docs/nodes-and-clients/}} 

There exists various implementations for Ethereum client in different languages (\eg Go, Rust, JavaScript,\etc). ``Geth'' is the most widely used Ethereum client written in Go. 
There are three different types of Ethereum nodes: Light node, Fast-sync and Archive full node.

\textit{Light node} stores the headers of the blocks and can have a limited verification such as state root verification. This type of clients are useful for devices that cannot store huge data.
\textit{Fast-sync Full node} stores all blockchain data, and can participate in new block generation. Also all past states can be derived from the full node but it takes time to grab the past data.
\textit{Archival Full node} not only stores all blockchain data, but also index them as well so that historical states can be accessed quickly on demand. In this thesis we call Archival Full node a
full node. 

An Ethereum full node has a JSON-RPC API that gives the user chance to use the methods implemented by the client and read the Ethereum data off-chain. Some of the methods that are used in this paper is listed below:

\begin{itemize}
    \item \emph{trace\_block}: Returns the transaction traces of all transactions in a specific block.
    \item \emph{eth\_getStorageAt}: Returns the value stored in a specific slot of a determined address. 
    \item \emph{eth\_getCode}: Returns the bytecode stored for the specific account address. If the address is EOA, it will return 0x0.
\end{itemize}

\section{Ethereum Use-cases}

In the previous section we mostly discussed the infrastructure of Ethereum and give the details about how Ethereum works under the hood and also how users can work with it. 
This section is mostly talk about the use-cases and nuances regarding them in Ethereum. Before explaining the use-cases one by one, we should shed light into one of the main obstacles that blockchain systems such as Ethereum has for developing different applications.


\paragraph{The Oracle Problem.}  

Smart contracts cannot access external resources (\eg a website or an online database) to fetch data that resides outside of the blockchain (\eg a price quote of an asset). External data needs to be relayed to smart contracts with an oracle. An \emph{oracle} is a bridge or gateway that connects the off-chain real world knowledge and the on-chain blockchain network. The `oracle problem'~\cite{linkOracleProblem} describes the limitation with which the types of applications that can execute solely within a fully decentralized, adversarial environment like Ethereum. Generally speaking, a public blockchain environment is chosen to avoid dependencies on a single (or a small set) of trusted parties. One of the first oracle implementations used a smart contract in the form of a database (\ie mapping\footnote{A Solidity \texttt{mapping} is simply a key-value database stored on a smart contract.}) and was updated by a trusted entity known as the \texttt{owner}. More modern oracle updating methods use consensus protocol with multiple data feeds or polling techniques based on the ``wisdom of the crowd''. The data reported by an oracle will always introduce a time lag from the data source and more complex polling methods generally imply longer latency.

\paragraph{Trusted Third Parties.} A natural question for smart contract developers to ask is: if you trust the oracle, why not just have it compute everything? There are a few answers to this question: (1) there may be benefits to minimizing the trust (\ie to just providing data instead of full execution), (2) there are widely trusted organizations and institutes---convincing one to operate an oracle service is a much lower technical ask than convincing one to operate a complete platform, and (3) if a data source becomes untrustworthy, it may require less effort to switch oracles than to redeploy the system. 

To mitigate the problem stated above, different solutions are developed to answer the oracle problem and provide data for the applications in Ethereum that needs the real-world data. We will discuss the problem and solution in detail in Chapter~\ref{ch:oracle}.

There are myriad of applications developed on top of Ethereum blockchain. Some of them uses the solutions describe in  Chapter~\ref{ch:oracle}, using one of them to address the oracle problem and bring real-world data on-chain. In later part of this section, we will explain the most favorable use-cases and applications on Ethereum.

\subsection{Stablecoins/Synthetic Assets}

A ``synthetic asset'' is an asset that tracks the price of another asset without holding the obligations of that asset. For instance, a synthetic asset of Apple share tracks the price of Apple share that does not receive dividends and not have any other obligations regarding the Apple share. This can be done by having the price of the asset provided by an oracle which is discussed above and some other ad-hoc mechanisms to stabilize the price of the asset that will be explained in Chapter~\ref{ch:dai}. An example Dapp that is providing synthetic assets is \textit{Synthetix}\footnote{\url{https://synthetix.io}}.

The asset can also be a currency (\eg USD) and in this case, it will be a stablecoin. A stablecoin is an asset that supposed to be peg to a currency such as USD. There are a myriad of stablecoin platform in Ethereum such as \dai which we will discuss in Chapter~\ref{ch:dai}

\subsection{Decentralized Exchange (DeX)}
Decentralized Exchanges (DeX) are platforms that helps users to exchange their assets without need of an intermediary. Two main types of Decentralized Exchanges are Automated Market Maker (AMM) and orderbook-based exchanges. 

In AMMs a party named Liquidity Provider (LP), creates pools for pairs of assets (could be more than two assets in a pool like Balancer protocol), and put liquidity on the pool. Users can exchange their assets in the pool. The smart contract will calculate the amount a user receive for a trade, using a mathematic formula based on the amount of liquidity in the pool and the amount user wants to exchange. Because the calculations and market making is done by smart contract automatically, this type of DeX is called Automated Market Maker. Uniswap\footnote{\url{https://uniswap.org}} is the most famous AMM running on Ethereum blockchain.

In traditional finance, majority of exchanges are orderbook-based. In orderbook-based exchanges, a user put her order to sell specific amount of asset A for specific amount of asset B. On the other hand, another user puts an order in reverse, to sell asset B for specific amount of asset A. The exchange will sort the orders of both sides and match the orders. This process is computational costly because of need of sorting and match making and not rational to be implemented on Ethereum mainnet because computations is expensive. But, there are platforms such as Loopring,\footnote{\url{https://loopring.org}} that implemented the whole process of order taking and match making off-chain, using a layer two solution named StarkEx,\footnote{\url{https://starkware.co/starkex/}} and put the results on-chain.

\subsection{Lending}
In lending platforms, users can lend and borrow crypto assets. There are pools of assets in these platforms that a user can lend those assets by depositing into the pools and can borrow against them. The borrowers are obliged to pay fees which is directly be paid to the lenders. Compound\footnote{\url{https://compound.finance}} and Aave\footnote{\url{https://aave.com}} are two most favorable lending Dapps on Ethereum at the time of writing the paper. 

\subsection{Ethereum Name Service (ENS)}
Ethereum Name Service (ENS)\footnote{\url{https://ens.domains}} is one of the most favorite Dapps in Ethereum that is not for financial purposes. ENS is a decentralized naming system that maps a human-readable names to Ethereum addresses. 
ENS recently adds support for crypto addresses other than Ethereum. Also users can launch their own websites using their ENS.

\subsection{Derivatives}
Derivatives are financial contracts that is based on value of another underlying asset or a basket of assets which include options, swaps, and future contracts. dYdX\footnote{\url{https://dydx.exchange}} is a decentralized trading platform that traders can go long or short up to 25x on specific assets like ETH, BTC, Link, \etc
Having 25x leverage position means that the user have exposure of 25x on the price changes means that if the user longs 25x, and if the price increase by 1\%, the user gains 25\%. If the user takes 25x short position and the price falls 1\% then the user gains 25\%.

Options are another type of derivatives in which the user has the right to buy\footnote{Call option} or sell\footnote{Put option} a specific amount of an asset, for a specific price.\footnote{strike price} Options has a expiration date: in American options style, one can exercise the contract before expiration date whenever they want but in European style the user can just exercise on the expiration date. Opyn\footnote{\url{https://www.opyn.co}} and Hegic\footnote{\url{https://www.hegic.co}} protocols are two options platform in Ethereum.


\subsection{Yield Farming}
Protocols like Compound give their native tokens to the active users. For instance, in Compound lenders and borrowers receive Comp token.\footnote{Native token of Compound} The aggregated fees collected by users in a year counted as Annual Percentage Yield (APY) which is completely related to the price of native token. There are different platforms that do almost the same logic, but with different APRs at the moment. This brings the idea to develop Dapps that calculate the best APR and use different strategies to maximize the APR of the user. Yearn\footnote{\url{https://yearn.finance}} is one of the leading Dapps that build different strategies to farm most benefit yields for the users. There are other yield farming Dapps such as Harvest finance\footnote{\url{https://harvest.finance}} or Pickle finance.\footnote{\url{https://www.pickle.finance}}

\subsection{Privacy Tools}
Ethereum blockchain do not ensure the anonymity and privacy on the main layer because of transparency. There are different solutions developed to bring privacy to the Ethereum transactions. One of the most favorite solutions is Tornado Cash.\footnote{\url{https://tornado.cash}} Tornado cash uses Zero-knowledge snarks to mix Ethers deposited into the contract and break the link between the deposited address and the withdrawn one. Another Dapp that uses Zero-knowledge to enable private transactions on Ethereum is Aztec protocol.\footnote{\url{https://aztec.network}}




\subsection{Liquidation} 
Liquidation is not a type of Dapp but because bunch of applications and use-cases described above have this process in common we put a section for it. It will be discussed in detail in Chapter~\ref{ch:dai}. Liquidation is used in stablecoins, synthetic assets, lending platforms, and derivatives. Liquidation happens when the value of the debt of a user exceeds a pre-specified portion of collateral provided by the user. So it means the user's collateral cannot back the debt in this situation. In traditional finance, the liquidation mostly managed by the broker or service provider. However, in DeFi, because of elimination of the intermediaries, the protocol incentivize the external actors, named \textit{Keepers}, to liquidate the position and top up the collateral or close buy the position.
\chapter{Not so immutable: Upgradeability of Smart Contracts on Ethereum} 
\label{ch:upgrade}

This chapter summarizes and evaluates six patterns, developed on Ethereum to enable upgradeability of smart contracts. Modern smart contracts use software tricks to enable upgradeability, raising the research questions of \textit{how} upgradeability is achieved and \textit{who} is authorized to make changes. We develop a measurement framework for finding how many upgradeable contracts are on Ethereum that use certain prominent upgrade patters. We also measure how they implement access control over their upgradeability: about 50\% are are controlled by a single Externally Owned Address (EOA), and about 20\% are controlled by multi-signature wallets in which a limited number of persons can change the whole logic of the contract which is a great risk to the Ethereum ecosystem.

\section{Introductory remarks}
The key promise of a smart contract running on Ethereum is that its code will execute exactly as it is written, and the code that is written can never be changed. While Ethereum cannot maintain this promise unconditionally, its assumptions (e.g, cryptographic primitives are secure and well-intentioned participants outweigh malicious ones) provide a realistic level of assurance. 

The immutability of a smart contract's code is related to trust. If Alice can validate the code of a contract, she can trust her money to it and not be surprised by its behavior. Unfortunately, disguising malicious behavior in innocuous-looking code is possible (``rug pulls''), and many blockchain users have been victims. On the other hand, if the smart contract is long-standing with lots of attention, and security assessments from third-party professional auditors, the immutability of the code can add confidence. 

The flip-side of immutability is that it prevents software updates. Consider the case where a security vulnerability in the code of a smart contract is discovered. Less urgently, some software projects may want to roll out new features, which is also blocked by immutability. There is an intense debate about whether this is a positive or negative, with many claiming that ``upgradeability is a bug.''~\footnote{\href{https://medium.com/consensys-diligence/upgradeability-is-a-bug-dba0203152ce}{``Upgradeability Is a Bug'', Steve Marx, Medium, Feb 2019.}} We do not take a position on this debate. We note that upgradeability is happening and we seek to study what is already being done and what is possible. 

Is there a way to deploy upgradeable smart contracts if all smart contracts are (practically speaking) immutable? Consider a two simple ideas. The first is to deploy the upgraded smart contract at a new address. One main drawback to this is that all software and websites need to update their addresses. A second simple idea is to use a proxy contract (call it P) that stores the address of the ``real'' contract (call it A). Users consider the system to deployed at P (and might not even be aware it is proxy). When a function is called on P, it is forwarded to A. When an upgrade is deployed to a new address (call it B), the address in P is changed from A to B. This solution also has drawbacks. For example, if the proxy contract hardcodes the list of functions that might be called on A, new functions cannot be added to B. Another issue is that the data (contract state) is stored in A. For most applications, a snapshot of A's state will need to be copied to B without creating race conditions. Mitigating these issues leads to more elaborate solutions like splitting up a contract logic and state, utilizing Ethereum-specific tricks (fallback functions to capture unexpected function names), and trying to reduce the gas costs of indirection between contracts.

\section{Contributions and Related Work} 
The state of smart contract upgradeability methods in Ethereum is mainly discussed in non-academic, technical blog posts~\cite{openzeppelinPost,tobBlogPost}. In Section~\ref{sec:classification}, we systemize the different types using these resources, and provide a novel evaluation framework for comparing them.

Fr{\"o}wis and B{\"o}hme~\cite{frowisnot} conducted a measurement study on the use-cases of the \texttt{CREATE2} opcode in Ethereum blockchain, which one of them is the Metamorphosis upgradeability pattern discussed in Section~\ref{sec:metamorphic}. They also find, in a passing footnote, some delegate-call based contracts by assuming compliance with the standards: EIP-897, EIP-1167, EIP-1822, and EIP-1967. In this chapter, we contribute a more general pattern-based measurement that is not specific to a standard or a commonly-used implementation. We also are the first, to our knowledge, to study who is authorized for upgrading an upgradeable contracts, shedding light on the risks of different admin types.

Recent papers have provided security tools for developers that compose with upgradeability patterns based on  \texttt{DELEGATECALL}~\cite{rodler2021evmpatch,perez2022dissimilar}. Numerous measurement studies have used Ethereum blockchain data but concern aspects other than upgradeability~\cite{perez2019broken,chen2017adaptive,reijsbergen2021transaction,victor2019measuring,pinna2019massive,he2020characterizing}. Chen et al.~\cite{chen2021smart} survey use-cases of the \textit{SELFDESTRUCT} opcode, but they do not cover how it is used in Metamorphosis~\ref{sec:metamorphic}.

\section{Classification of Upgrade Patterns} \label{sec:classification}

\paragraph{Updating vs. upgrading.} Software maintenance is part of software's lifecycle, and the process of changing the product after delivery. Often a distinction is drawn between software \textit{updates} and software \textit{upgrades}. An update modifies isolated portions of the software to fix bugs and vulnerabilities. An upgrade is generally a larger overhaul of the software with significant changes to features and capabilities. We only use the term upgrade and distinguish between retail (parameters and isolated code) and wholesale (entire application) changes to a smart contract. While upgrades to a smart contract's user interface (UI) can significantly change a user experience and expose new features, UIs are governed by traditional software maintenance. This chapter only considers the on-chain smart contract component, which is significantly more challenging to upgrade as it is on-chain and immutable under reasonable circumstances.

\begin{figure}[t]
  \centering
      \includegraphics[width=0.8\textwidth]{figures/New_Classification.png}
  \caption{Classification of upgradeability patterns.\label{fig:class}}
\end{figure}

A variety of upgradeability patterns have been proposed for smart contracts. Most leverage Ethereum-specific operations and memory layouts and are not applicable to other blockchain systems.


\subsection{Parameter Configuration}
\label{sec:parameter}

We first categorize upgradeability patterns into two main classes: \textit{retail changes} and \textit{wholesale changes}. A pattern for retail change does not enable the replacement of the entire contract. Rather, a component of the contract is pre-determined (before the contract is deployed on Ethereum) to allow future upgrades, and the code is adjusted to allow these changes. 

The simplest upgrade pattern is to allow a system parameter, that is stored in a state variable, to be changed. This requires a \textit{setter function} to overwrite (or otherwise adjust) the variable, and access control over who can invoke the function. For example, in decentralized finance (DeFi), many services have parameters that control fees, interest rates, liquidation levels, \etc Adjustments to these parameters can initiate large changes in how the service is used (its `tokenomics'). A DeFi provider can retain control over these parameters, democratize control to a set of token holders (\eg stability fees in the stablecoin project MakerDao), or lock the parameters from anyone's control. In Section~\ref{sec:governance}, we dive deeper into the question who can upgrade a contract. 

% = = =

\subsection{Functional Component Change}
\label{sec:component}

While a parameter change allows an authorized user to overwrite memory, a functional component change addresses modifications to the code of a function (and thus, the logic of the contract). In the EVM, code cannot be modified once written and so new code must be deployed to a new contract, but can be arranged to be called from the original contract. 

One way to allow upgradable functions is deploying a helper contract that contains the code for the functions to be upgradeable. Users are given the address of the primary contract, and the address of this secondary contract is stored as a variable in the primary contract. Whenever this function is invoked at the primary contract, the primary contract is pre-programmed to forward the function call, using the opcode \texttt{Call}, to the address it has stored for the secondary contract. To modify the logic of the function, a new secondary contract is deployed at a new address, and an authorized set of individuals can then use a parameter change in the primary contract to update the address of the secondary contract.

The DeFi lending platform Compound~\footnote{\url{https://compound.finance}} uses this pattern for their interest rate models~\footnote{\url{https://github.com/compound-finance/compound-protocol/blob/v2.3/contracts/InterestRateModel.sol}} which are tailored specifically for each asset. The model for one asset can be changed without impacting the rest of the contract~\cite{openzeppelinPost}.

Upgradeable functional components need to be pre-determined before deploying the primary contract. Once the primary contract is deployed, it is not possible to add upgradeability to existing functions. It also cannot be directly used to add new functions to a contract. Finally, this pattern is most straightforward when the primary contract only uses the return value from the function to modify its own state. Thus, the function is either `pure' (relies only on the parameters to determine the output) or `view' (can read state from itself or other contracts, but cannot write state). If the function modifies the state of the primary contract, the primary contract must either expose its state variables to the secondary contract (by implementing setter functions), or it can run the function using \texttt{Delgatecall} if the secondary contract has no state of its own. 

This upgrade pattern suggests a way forward for wholesale changes to the entire contract: create a generic ``proxy'' contract that forwards all functions to a secondary contract. To work seamlessly, this requires some further engineering (Sections~\ref{sec:callbased} and \ref{sec:delegatecall}).


\subsection{Consensus Override}
\label{sec:hardfork}

The two previous patterns enable portions of a smart contract to be modified. The remaining patterns strive to allow an entire contract to be modified or, more simply, replaced. The first wholesale pattern is not a tenable solution to upgradeability as it as only been used rarely under extraordinary circumstances, but we include it for completeness. 

Immutability is enforced by the consensus of the blockchain network. If participating nodes (\eg miners) agreed to suspend immutability, they can in theory allow changes to a contract's logic and/or state. If agreement is not unanimous, the blockchain can be forked into two systems---one with the change and one without. In 2016, a significant security breach of a decentralized application called ``the DAO'' caused the Ethereum Foundation to propose overriding the immutability of this particular smart contract to reverse the impacts of attack. In the unusual circumstances of this case, it was possible to propose and deploy the fix before the stolen ETH could be extracted from the contract and circulated. Nodes with a philosophical objection to overriding immutability continued operating, without deploying the fix, under the name Ethereum Classic.

% = = =

\subsection{Contract Migration}
\label{sec:migration}

The simplest wholesale upgrade pattern is to deploy a new version of the contract at a new address, and then inform users to use the new version---called a ``social upgrade.'' One example is Uniswap\footnote{\url{https://uniswap.org}}, which is on version 3 at the time of writing. Versions 1 and 2 are still operable at their original addresses. 

Contract migration does not require developers to instrument their contracts with any new logic to support upgradeability, as in many of the remaining patterns, which can ease auditability and gas costs for using the contract. However for most applications, there will be a need to transfer the data stored in the old contract to the new one. This is generally done in one of two ways. The first is to collect the state of the old contract off-chain and load it into the new contract (\eg via its constructor). If the old contract was instrumented with an ability to pause it, this can eliminate race-conditions that could otherwise be problematic during the data migration phase. The second method, specific to certain applications like tracking a user's balance of tokens, is to have the user initiate (and pay the gas) for a transfer of their balance to the new contract.
 
 % = = =

\subsection{\texttt{CREATE2}-based Metamorphosis}
\label{sec:metamorphic}

Is it possible to do contract migration, but deploy the new contract to the \textit{same} address as the original contract, effectively overwriting it? If so, developers can dispense with the need for a social upgrade (but would still need to accomplish data migration). At first glance, this should not be possible on Ethereum, however a set of opcodes can be ``abused'' to allow it: specifically, the controversial\footnote{\href{https://www.reddit.com/r/ethereum/comments/lx32kv/expectations\_for\_backwardsincompatible\_changes/}{``Expectations for backwards-incompatible changes / removal of features that may come soon.'' V. Buterin, Reddit r/ethereum, Mar 2021.}} \texttt{SELFDESTRUCT} opcode and the 2019-deployed \texttt{CREATE2}. 

Consider a contract, called Factory, that has the bytecode of another contract, A, that Factory wants to deploy at A's own address. \texttt{CREATE2}, which supplements the original opcode \texttt{CREATE}, provides the ability for Factory to do this and know in advance what address will be assigned to contract A, invariant to when and how many other contracts that Factory might deploy.  The address is a structured hash of A's ``initialization'' bytecode, parameters passed to this code, the factory contract's address, and a salt value chosen by the factory contract.\footnote{Specifically: $\mathsf{addr} \leftarrow \mathcal{H}(\mathtt{0xff} \| \mathsf{factoryAddr} \| \mathsf{salt} \| \mathcal{H} (\mathsf{initBytecode} \| \mathsf{initBytecodeParams}))$} Most often, A's initialization bytecode contains a copy of A's actual code (``runtime'' bytecode) to be stored on the EVM, and the initialization code is prepended with a simple routine to copy the runtime code from the transaction data (calldata) into memory and return. Importantly, however, the initialization bytecode might not contain A's runtime bytecode at all, as long as it is able to fetch a copy of it from some location on the blockchain and load it into memory. In order for \texttt{CREATE2} to complete, the address must be empty, which means either (1) no contract has ever been deployed there, or (2) a contract was deployed but invoked \texttt{SELFDESTRUCT}.

Assume the developer wants to deploy contract A using metamorphosis and later update it to contract B.\footnote{\href{https://medium.com/@0age/the-promise-and-the-peril-of-metamorphic-contracts-9eb8b8413c5e}{``The Promise and the Peril of Metamorphic Contracts.'' 0age, Medium, Feb 2019.}} The developer first deploys a factory contract with a function that accepts A's (runtime) bytecode as a parameter (which includes the ability to self destruct). The factory then deploys A at an arbitrary address and stores the address in a variable called codeLocation. The factory then deploys a simple `transient' contract using \texttt{CREATE2} at address T. This contract performs a callback to the factory contract, asks for factory.codeLocation, and copies the code it finds there into its own storage for its runtime bytecode and returns. As a consequence, A's bytecode is now deployed at address T. 

To upgrade to contract B, the developer calls \texttt{SELFDESTRUCT} on A. Mechanically, the consequences of \texttt{SELFDESTRUCT} on the EVM are only realized at the end of the transaction. In a followup transaction, the developer calls the factory with contract B's bytecode. The factory executes the same way placing a pointer to B in factory.codeLocation. Importantly, it generates the same address T when it invokes \texttt{CREATE2} since the `transient' contract is identical to what it was the first time---this contract does not contain contract A or B's runtime code, it just contains abstract instructions on how to load code. The result is contract B's runtime bytecode being deployed at address T where contract A was. 
  

As it is concerning that a contract's code could completely change, we note that metamorphic upgrades can be ruled out for any contract where either: it was not created with \texttt{CREATE2}, it does not implement \texttt{SELFDESTRUCT}, and/or its constructor is not able to dynamically modify its runtime bytecode. 

% = = =

\subsection{\texttt{CALL}-based Data Separation}
\label{sec:callbased}

To avoid migrating the stored data from an old contract to an upgraded contract, a contract could instead store all of its data in an external ``storage'' contract. In this pattern, calls are made to a ``logic'' contract which implements the function (or reverts if the function is not defined). Whenever the logic contract needs to read or write data, it will call the storage contract using setter/getter (aka accessor/mutator) functions. An upgrade consists of (1) deploying a new logic contract, (2) pausing the storage contract, (3) granting the new logic contract access to the storage contract, (4) revoking access from the old contract, and (5) unpausing the storage contract. 

An important consideration is that the layout of the storage contract cannot be changed after deployment (\eg we cannot add a new state variable). This can be side-stepped to some extent by implementing a mapping (key-value pair) for each primitive data type. For example, a new uint state variable can be a new entry in the mapping for uints. This is called the Eternal Storage pattern (ERC930). It however requires that every data type be known in advance, and is challenging to use with complex types (\eg structs and mappings themselves).

A variant of this pattern can introduce a third kind of contract, called a proxy contract, to address the social upgrade problem. In this variant, users permanently use the address of the proxy contract and always make function calls to it. The proxy contract stores a pointer (that can be updated) to the most current logic contract, and asks the logic contract to run the function using \texttt{CALL}. Unlike the functional component pattern (Section~\ref{sec:component}), the proxy will catch and forward \textit{any} function (including new functions deployed in updated logic contracts) using its fallback function.  With or without proxies, this pattern is very powerful, but instrumenting a contract to use it requires deep-seated changes to the contract code. As our measurements will show, it has fallen out of favour for the cleaner \texttt{DELEGATECALL}-based pattern (Section~\ref{sec:delegatecall}) that addresses the same issues with simpler instrumentation. 


% = = =

\subsection{\texttt{DELEGATECALL}-based Data Separation}
\label{sec:delegatecall}

This pattern is a variant on the idea of chaining each function call through a sequence of three contracts: proxy, logic, and storage. The first modification is reversing the sequence of the logic and storage contracts: a function call is handled by the proxy which forwards it to the storage contract (instead of the logic contract). The storage contract then forwards it to the logic contract using \texttt{DELEGATECALL} which fetches the code of the function from the logic contract but (unlike \texttt{CALL}) runs it in the context of the contract making the call---\ie the storage contract. When upgrading, a new logic contract is deployed, the proxy still points to the same storage contract, and the storage contract points to the new logic contract. Since the proxy and storage contracts interact directly and are both permanent, the functionality of both can be combined into a single contract. It is common for developers to call this the `proxy contract,' despite it being a combination of a proxy and a storage contract. 

This pattern is generally cleaner than using the previous \texttt{CALL}-based pattern because the logic contract does not need any instrumentation added to it. It is an exact copy of what the contract would look like if the upgrade pattern was not being used at all. However this does not mean the pattern in a turn-key solution. Each new logic contract needs to be programmed to respect the existing memory layout of the storage contract, which has evolved over the use of all the previous logic contracts. The logic contract also needs to be aware of any functions implemented by the storage contract itself---if the same function exists in both the storage contract and the logic contract (called a function clash), the storage function will take precedence.


The main issue with function clashes is that the proxy contract needs, at the very least, to provide an admin (or set of authorized parties) the ability to change the address of the logic contract it delegates to. This can be addressed in four main ways:

\begin{enumerate}

\item Developers are diligent that no function signature in the logic contract is equal to the signature of the upgrade function in the proxy contract (note that signatures incorporate a truncated hash of the function name, along with the parameters types, so collisions are possible). 

\item As found in the \emph{universal upgradeable proxy standard (UUPS)} (EIP-1822): implement the upgrade function in the logic contract, which will run in the context of the proxy contract. Its exact function signature must be hardcoded into the proxy contract. Every logic contract update must include it or further updates are impossible.

\item As found in the \textit{beacon proxy} pattern (EIP-1967): deploy another contract, called the beacon contract, to hold the address of the logic contract and implement the setter function for it. The proxy contract will get the logic contract address from the beacon every time it does a \texttt{DELEGATECALL}. The admin calls the beacon contract to upgrade the logic contract, while normal users call the proxy contract to use the DApp. 

\item As found in the \textit{transparent proxy} pattern (EIP-1538): inspect who is calling the proxy contract (using \texttt{msg.sender()})---if it is the admin, the proxy contract catches the function call and if it is anyone else, it is passed to the proxy's fallback function for delegation to the logic contract. 

\end{enumerate}

A drawback of the entire \texttt{DELEGATECALL}-based pattern is that logic contracts need to be aware of the storage layout of the proxy contract. In a stand-alone contract, the compiler (\eg Solidity) will allocate state variables to storage locations, and using \texttt{DELEGATECALL} does not change that, however new logic contracts need to allocate the same variables in the same order as the old contract, even if the variables are not used anymore. This can be made easier with object-oriented patterns: each new logic contract extends the old contract (inheritance-based storage). Other options include mappings for each variable type (eternal storage) or hashing into unique memory slots (unstructured storage). The \textit{Diamond Storage} pattern (EIP-2535) breaks the logic contract into smaller clusters of one or a few functions that can be updated independently, and each can request one or more storage slots in a storage space managed by the proxy contract itself. 


\section{Evaluation Framework}
\label{app:eval}

\input{chapters/chapter3/table}

In this section, we compare and evaluate different methods discussed in previous section and explain the consequences regarding each method to the users and developers of Dapps.
Table~\ref{tab:eval} summarizes the pros and cons of each upgradeability pattern, omitting consensus override as it is only used in emergencies. The detail of criteria and properties that are used to assess each method, and also take-aways from the evaluation are described in further sections.

\subsection{Properties}

There are some characteristics that can help the designer to decide which method should be used on the system and add upgradeability to the Dapp. In this part we pencil out these criteria and evaluate different methods based on these criteria. In this part we describe and specify what it means that each row of our table receives a full dot (\full), partial dot (\prt), or nothing. 

\subsubsection{Can replace entire logic}
An upgradeability method in which the admin is able to replace the entire logic of the system earns a full dot (\full) otherwise it receives nothing.

\subsubsection{No need to migrate state from old contract}
In some patterns, there is no need to collect data from the old version and push it to the new contract which receive a full dot (\full). On the other hand, patterns which required to migrate data from old version receive nothing.

\subsubsection{User endpoint address not changed}
In some upgradeability methods, after the upgrade process, users must call a new contract address to use the Dapp. It is equivalent to having 2 different Dapps at the end of the upgrade. Alice uses Dapp X which uses one of the upgradeability patterns at address A before the upgrade. After upgrade, she may be unaware that upgrade happened and use the previous address (receive full mark (\full)) or she may need to use address B instead which receive nothing.

\subsubsection{No need to instrument source}
Upgradeability patterns in which the developers do not need to change any part of the original code to add the upgrade method receives nothing. The methods in which the developers do not need to change the whole code but should add a proxy contract or change just one component of the system receive half dot (\prt) and patterns in which the developers should change the whole code to add upgradeability receive full mark (\full).


\subsubsection{No need to deploy a new contract}
In some upgradeability patterns, the admin needs to deploy a new smart contract in the process of upgrade which receives nothing. Upgradeability methods which do not need to deploy a new contract for the process of upgrade receive a full dot (\full).

\subsubsection{No indirection between contracts}  
Indirection happens if the first external message need be forwarded from a contract to another using one of the \texttt{CALL}, \texttt{STATICCALL}, or \texttt{DELEGATECALL} opcodes. Upgradeability methods that do not need any indirections receive a full dot (\full). An upgradeability pattern that contains indirections which adds an extra gas because of adding one or more layers of indirection awarded nothing. An upgradeability method in which not all but just a portion of the incoming transactions need indirection receive half dot (\prt). 


\subsubsection{No downtime to upgrade}
Patterns which have a downtime of the Dapp in the upgrade event receive full dot (\full) otherwise it receives nothing. 

\subsubsection{Function Selector Clashes}
Upgradeability methods in which the developer should take care of function selector risks due to the using of \texttt{DELEGATECALL} opcode receive full dot (\full), otherwise receive nothing. 

\subsubsection{Storage Clashes}
Upgradeability methods in which the developer should take care of storage clashes risks in two contracts due to the using of \texttt{DELEGATECALL} opcode receive full dot (\full). 



%%%%%%%%%%%%%%%%%%%%%%%%%%%%%%%%%%%%%%%%%%%%%%%%%%%%%%%%%%%%%%%%%%%%%%%%%%%%%%%%%%%%%%%%%%%%%%%%%%%
%%%%%%%%%%%%%%%%%%%%%%%%%%%%%%%%%%%%%%%%%%%%%%%%%%%%%%%%%%%%%%%%%%%%%%%%%%%%%%%%%%%%%%%%%%%%%%%%%%%

\subsection{Evaluation Framework: Take-Aways}
\label{app:eval2}

In this section we discuss about the consequence of each upgrade methods regarding the criteria we mentioned in the previous part for users and developers that want to use the upgradeability pattern or uses a Dapp that uses one of the mentioned patterns.

\subsection{Speed of an Upgrade}
Upgrade events of a Dapp consists of two different processes. First a way to come to an agreement for changing the system, and then a way to implement and execute the change. The First part depends on the reason behind the upgrade. If the upgrade is to patch a bug, then the process to come into agreement is very fast but if the goal behind upgrade is to add new functionality or change a logic, it usually starts with a proposal and after some discussions, if the agent that responsible for the decision agree with the proposal, the execution part will be started. We won't discuss the first process because it depends on the type of agent discuss in \ref{sec:governance}.
After coming into agreement about the change, the speed that the admin can implement and execute the upgrade depends on three criteria discussed above: \textit{No need to deploy a new contract to upgrade}, \textit{No need to migrate state from old contract}, and \textit{having a downtime in the upgrade process}.

\textit{Parameter change} method is the fastest way to execute the upgrade because there is no need to deploy a new contract, and no need to migrate state and no downtime in the system.
\textit{Component change} method change is not as fast as parameter change method but faster than other types because the admin needs to deploy a specific smart contract which is a small component of the system and also update an address variable inside the main contract that points to that specific component and change it to the address of the new version of that component. But there is no need to migrate data and there is no downtime needed for this upgrade method.

\textit{Migration} method has a slow upgrade process. The reason is that the admin needs to deploy a new contract and also the admin or users should transfer the data from old version to the newer version. In most migration processes the developer team deploy a \textit{Migrator} contract and users should use this Migrator contract to withdraw their funds/data from the previous version and move it to the newer version. But, there is no downtime in the Dapp and no need to change a state variable.

\textit{Call-based} and \textit{DelegateCall-based} are very similar to each other in the speed of upgrade. These two are not as quick as \textit{Retail changes} because the developer needs to implement and deploy the \textit{whole logic} contract to the blockchain and then change the pointer addresses inside the storage/proxy contract to the newer version.
On the other hand these two approaches are faster than \textit{Migration} because as mentioned before, there is no need to migrate data. There is no downtime in these methods.

\textit{Metamorphic} method is the slowest way to upgrade a system which uses this method because similar to the migration plan there is a need to deploy a contract and migrate the state to the newer version but there is a difference between these two. In metamorphic method, the admin first should \textit{Self-Destruct} the previous version in a single transaction and after that transaction send a contract creation transaction to deploy the newer version. Because self-destruct happened at the end of the transaction, the process of upgrade happens on two different transactions which is a downtime to the system. This downtime could be a gap between order of the two transaction in a single block or could be gap between blocks that these two transactions included into blockchain. 


\subsection{Cost of Upgrade}

One of the main differences between upgradeability approaches is how much does the upgrade process costs for the admins and users. The cost of upgrade mostly depends on two criteria explained above: 1) no need to deploy a new contract, 2) no need to migrate a the state to newer version.

\textit{Parameter change} method is the cheapest method in the upgrade event because there is no need to deploy a new contract or migrate data.
\textit{Component change} is in the middle, because there is a need to deploy a new contract (however it is cheaper comparing to methods in which we should deploy the whole logic), but there is no need for data migration.
\textit{Migration} plan is very expensive in the upgrade event because we need to deploy a new contract and migrate the data from the old version which is very expensive.
\textit{Call-based} and \textit{DelegateCall-based} are very similar to each other in the cost of upgrade which is more expensive than component change, but cheaper than migration. In both the admin must deploy the a contract containing the whole logic, but there is no need to migrate the whole data.
\textit{Metamorphic} method is the most expensive method we have because we need to deploy a new contract, migrate data to the newer version and also we need to self-destruct the previous version before the upgrade event which adds cost to the upgrade process.


\subsection{Gas overhead for users}
Sometimes in upgradeability patterns, we have a tradeoff between adding a feature to the pattern to improve it and increasing the cost (gas needed for the transactions) for users that want to interact with our Dapp.

In patterns that needs indirection, such as \textit{Call-based}, \textit{Delegatecall-based}, and \textit{Component change} pattern an extra cost will be added to the users, because for all or some of the transactions to the Dapp, our system needs to forward the calls to another contract using Call or Delegatecall opcode to the users. 
Also in \textit{Delegatecall-based} pattern to mitigate the function selector clashes or storage clashes, we need to add some other checks to our code which also increases the cost of interacting with the Dapp.
Also there are some other ideas that addresses some limitations of one type of upgradeability pattern, but increases the cost for users. For instance, in \textit{Call-based} approach one of the problems is that after upgrade users should use a new address for using the Dapp but adding a \emph{Registry} contract can help to mitigate this. Using Registry contract, all other contracts should ask the registry to find out the latest version of the contract and then calls to the newer version which adds a gas cost to the users. 

\subsection{Useability}
Upgradeability patterns differ in term of Useability and it depends on three criteria explained above; \textit{User endpoint address unchanged}, \textit{No need to migrate state from old contract} and \textit{Downtime in upgrade events}.

Patterns in which the endpoint address is changing after upgrade event, \textit{Migration} and \textit{Call-based} is not user friendly because each time that the upgrade happens, the user must use the newer address. So there is need to make awareness about this address change which is a hard action and need to socially interact with the all users and make them aware of the change. We have two main type of users in the Dapp ecosystem, normal user or another smart contract (Dapp) that uses our system. Regular users which uses the official interface (website) of the project may do not sense any changes, but users that work with the smart contract directly or via their own interface (e.g, Centralized Exchanges), or other Dapps that uses the smart contract must have a way to upgrade the address they uses to use the newer version and if they did not implement a way to upgrade this address then their Dapp will face problems. So these patterns are make problems for composability of the ecosystem.

In most of \textit{Migration} plan upgrade events, users are responsible for the migration of their data using a migrator contract (for instance, the user must withdraw the fund and use a migrator contract to push the data into the newer version) which add costs to the user and it is not user friendly. This is one reason that make the migration plans very hard because some users are not doing the process of migration and stay on the previous version which is like having a fork for the Dapp in side of the Dapp team (e.g, Uniswap V2 and V3).
In \textit{Metamorphic} pattern as mentioned before there is a downtime during the upgrade. So users cannot work with the Dapp on that exact time which is not user friendly.


\subsection{Dealing with two different new versions}
In \textit{Migration} and \textit{Call-based} pattern we will come up with two different Dapps after each upgrade event. So, a decision must be made for the previous version. One possible choice could be shutting down the old version. It can be done by self-destructing the old version, or by having a pausing mechanism to stop the older version functionality. In migration plan it is not regular to stop the previous version because in most migration plans, users are responsible to move their funds and data from the previous version to the new one and we cannot force them to do that, so we cannot stop the smart contract. 
The other option could be having a mechanism that after the upgrade, all calls to the previous version just be forwarded to the newer version which add costs and have some limitations like we cannot call the new functions defined in the newer version using the old version. This option is doable in Call based patterns. The other problem of this option is that if we upgrade a system more than one time then the calls to the first version should be redirected through lots of contracts to reach to the newer version. Also it adds complexity because developers must maintain more than one contract~\cite{tobBlogPost}.


\subsection{System Complexity} \label{sysComplexity}

Using upgradeability patterns will add to complexity of our system but the degree of complexity varies and depends on the pattern. 
\textit{Parameter Change} method does not change the system in general but just adding a mechanism to change pre-specified variables in the system. The most important issue about this pattern is that the developer team must limit the boundary of these parameter for the security of the system. For instance in MakerDao platform~\footnote{\url{https://makerdao.com/}}, Stability fee is changeable but if this variable be changed to 100\% then the whole system will be halted, so it should be limited.
\textit{Component Change} pattern is very similar to the parameter change, but a whole component could be changed and finding the safe boundary of changes and limiting this boundary is a bit harder.
\textit{Migration} plans for upgradeability does not change any complexity to the system because we do not need to change any part of system to add this type of upgradeability to it. The only important issue regarding this pattern is that we must be sure that there is a way to collect data from the old version like having getter functions for reading data and also having a withdraw function for users to collect data and funds from previous version and push or deposit it to the newer version.
Using \textit{Call-based} patterns adds higher degree of complexity to the system compared to previous patterns. As discussed before, in this pattern we must be sure that the storage and logic contract is divided and there is not any storage variable inside the logic contract. This is one of the main security issues that found in the Dapps using this pattern regarding Trail of Bits company reports~\cite{tobBlogPost}. To add a way in storage contract to define new variables, developers uses the eternal Storage pattern for their storage contract which is very hard to apply for complex data structures in Ethereum such as mappings or structures. This is another source of complexity using Call-based pattern.
\textit{Delegate-call} pattern adds complexity to the code because of using \textit{Delegate-call} opcode in its logic. As mentioned above because of using this opcode, the developer should take care of storage clashes and also function selector clashes. Other than these two there are some other limitations and risks of using this patterns. For instance, we cannot have a \textit{Constructor} function on the logic contract, because constructor functions is used to initialize specific variables at deployment time and if we have a constructor inside the logic, then storage of implementation contract will be changed and not storage of proxy contract. To mitigate this problem we can add a regular function named \textit{Initialize} function inside the implementation and make sure that this function can be called \emph{once} to act just like a constructor function. 
\textit{Metamorphic} pattern is proposed recently and not well-tested yet. There are some risks to this pattern as well. We should be sure that we have a mechanism to self-destruct the contract. Otherwise we cannot redeploy a new version and so our contract won't be upgradeable. The other important issue related to Metamorphic pattern is that the developer must know that each time they want to upgrade the system the whole storage will be wiped out and need to re-initiate the whole state after re-deployment. 

%%%%%%%%%%%%%%%%%%%%%%%%
%%%%%%%%%%%%%%%%%%%%%%%%

\section{Finding Upgradeable Contracts} 
\label{sec:proxyFinding}

We now design a series of measurement studies to shed light on the prevalence of the various upgrade patterns. We exclude retail changes from our measurements, because variable changes and external function calls are too commonplace to distinguish. We focus on wholesale patterns, and devote the most effort to finding contracts using the \texttt{DELEGATECALL}-based data separation pattern (Section~\ref{sec:delegatecall}) as these are the most widely used and there are various sub-types (UUPS, beacon, \etc). The other types of wholesale patterns are: 
\begin{itemize}
\item \textbf{Consensus override:} Only 1 occurrence to date (the DAO attack~\cite{dhillon2017dao}).
\item \textbf{Contract migration:} Not detectable in code; relies on social communication of the new address.
\item \textbf{\texttt{CREATE2}-based metamorphosis.} Already measured by Frowis and Bohme~\cite{frowisnot} in a broader study of all uses of \texttt{CREATE2}. They found 41 contracts between March 2019 and July 2021 that upgraded using this pattern.
\item \textbf{\texttt{CALL}-based data separation.} We conducted a quick study of 93K contracts with disclosed source code~\cite{smart_contract_sanctuary}. We identified the Eternal Storage pattern using regular expressions and found 140 instances, the newest having been deployed over 3.5 years old. We conclude this pattern is too uncommon today to pursue a deeper bytecode-based on-chain measurement.  
\end{itemize}


\subsection{Methodology} 

\begin{figure}[t!]
 \includegraphics[width=1\textwidth]{figures/New-method.png}\label{flowchart}
 \caption{Flowchart for distinguishing upgradeable contracts (green) from forwarders, and for determining the upgradeability pattern type.}
\end{figure}

% % %

\paragraph{Finding proxies.} While not every use of a proxy contract is for upgradeability (\eg minimal proxies~\cite{minimalProxy}, \texttt{DELEGATECALL} forwarders~\cite{delegatecallForwarders}, \etc), all \texttt{DELEGATECALL}-based upgradeability variants have the functionality of a proxy. We therefore start by measuring the number of contracts with a proxy component, and then filter out the \textit{Forwarders} which do not enable upgradeability. To identify proxies, we examine every \texttt{DELEGATECALL} action and see if it was proceeded by a call with an identical function selector to the contract making the \texttt{DELEGATECALL} action, which indicates   the contract does not implement this function and instead caught it in its fallback function, and is now forwarding it to another contract at, what we will call, the \textit{target address}. We used an Ethereum full archival node\footnote{\url{https://archivenode.io/}} and replayed each transaction in a block to obtain Parity VM transaction traces. \texttt{DELEGATECALL} is one \texttt{callType} of an \texttt{action} within a trace. Specifically, if the data of two consecutive actions of a transaction are equal and a \texttt{DELEGATECALL} is in the second action, it shows that the transaction passes the fallback function (if any other function in the contract is called, other than fallback, then the first four bytes of the data will be changed). The \texttt{DELEGATECALL} indicates the fallback transferred the whole data to the target address without altering it, which means the contract implements a proxy.

\paragraph{Distinguishing forwarders and upgradeability patterns.} In an upgradeable contract, the target address for the \texttt{DELEGATECALL} must be modifiable. If it is fixed, we tag it as a forwarder. We define five common patterns for determining the target address cannot be changed:

\begin{enumerate}
 \item The target address is hardcoded in the contract.
 \item The target address is saved in a constant variable type.
 \item The target address is saved in an immutable variable type and the deployer sets it in a constructor function.
 \item The target address is defined as an unchangeable storage variable.
 \item The proxy contract grabs the target address by calling another contract but there is no way the callee contract can change this address.
\end{enumerate}

In the first three situations, the target address will be appeared in the runtime bytecode of the contract. For every proxy-based \texttt{DELEGATECALL}, we obtain the target address from the transaction's \texttt{to address}, and we obtain the caller's bytecode by invoking \texttt{eth\_getCode} on the full node. If we find the target address in the bytecode, we mark it as a forwarder. 

In the fourth case, we find where the target address is stored by the contract by decompiling the contract, with \textit{Panoramix}\footnote{\url{https://github.com/palkeo/panoramix}}, locating the line of code in the fallback function that makes the \texttt{DELEGATECALL}, and marking the storage slot for the target address. We parse the code and check if an assignment to that slot happens in any function in the contract. The process is explained in Section~\ref{app:assignment} in detail. If any assignment is found, we should be sure that the other variable assigned to the target address variable comes from the input of that function. If these conditions are satisfied, there is a function inside the contract that can change the target address and we mark the proxy as an upgradeable proxy contract. 

Recall in the Universal Upgradeable Proxy Standard (UUPS) pattern, the logic contract implements a function to update the target address that is run in the proxy contract's context using \texttt{DELEGATECALL}. This is a sub-case of the fourth case, where we check the logic contract instead of the proxy contract. If we determine the logic contract can assign values to the logic contract in any function, we tag it as UUPS.

In the fifth case, we rewind the transaction trace from the proxy-based \texttt{DELEGATECALL} and look for the target address being returned to the proxy contract in another action. If we find it being returned by a contract, we apply the methodology from the fourth case to this contract. If the target address is modifiable, we mark it as using the Beacon proxy upgradeability pattern. All contracts that remain after performing all of the checks above are marked as forwarders. 


 %%%%%%%%%%%%%%%%%%%%%%%%%%%%%%%%%%%%%%%%%%%%%%%%%%%%%%%%%%%%%%%%%%%%%%%%%%%%%%%%%%%%%%%%%%%%%%%%%%%%%%%%%%%%%%%%%%%%%%%%%%%%%%%%%%%%%%
 \subsection{Assignment Checker Module}
\label{app:assignment}

\begin{figure}[t!]
 \centering
 \includegraphics[keepaspectratio,height=20cm, width=30cm]{figures/Upgradeability_finder-3.png}
 \caption{Upgradeability Proxy Contract Finder}
 \label{fig:finderModule}
\end{figure}

The whole measurement process is depicted on figure~\ref{fig:finderModule}. We need a module to check whether the admin can change target address on the proxy contract,using a function in the proxy contract, implementation contract or beacon contract. For this purpose the module must get the \textit{Bytecode} of the proxy,implementation or beacon address as input and find the variable name and also its storage slot of the target address. Then checks to find out is there any function inside the contract that gives the admin the ability to change the target address.

We use bytecode decompiler named \textit{Panoramix decompiler} \footnote{\url{https://github.com/palkeo/panoramix}} to decompile the bytecode into well-formatted python language codes. The decompiled code gives us all storage variables of the related contract and the storage slots of those variables in a function named \textbf{Storage}. On the other hand, the decompiled code will tell us if a function is \textit{Payable} or not. Among these Payable functions the one that does not have name or its name is fallback is the \textit{fallback} function of the contract. So we will try to find the line of code that \textit{Delegate Call} happened on it and collect these lines. Now that we have storage variable names and storage slots of these variables and also the line of code inside fallback that have the delegatecall, we will check to find the target address variables. We are doing that by checking if one of the storage variables inside Storage function is used in the line of code that contains delegate call. We will add them to an array of implementation addresses.

There is two other steps here. First finding other variable names with the same storage slots as the implementation addresses we found from the first step by checking the Storage function and also finding another variables that being assigned to those implementation variables in some other part of the code. We will add these two type of variables to the implementation addresses as well.

Now that we have a list for implementation addresses, we will search through the code to find if any assignment happened to one of them. If yes we will pick the variables that is assigned to target variable and then check if this assignment happened in a specific function and to one of the inputs of that function. In this case this function will be the upgrade function because the caller of this function can upgrade the target address by calling this function with desired input. 

To summarize what we did, we find all possible variables in the code that can change the target address inside the contract and check if there is any function inside them that can assign new address to the target address variable.

The whole process is depicted on figure~\ref{assignmentFinder}.

\begin{figure}[t]
 \centering
 \includegraphics[keepaspectratio,height=20cm, width=30cm]{figures/Assignment_finder.png}
 \caption{Assignment CheckerModule}
 \label{assignmentFinder}
\end{figure}

 %%%%%%%%%%%%%%%%%%%%%%%%%%%%%%%%%%%%%%%%%%%%%%%%%%%%%%%%%%%%%%%%%%%%%

 \subsection{Results}

 \begin{table}[t]
 \centering
 \begin{tabular}{|l|r|}
 \hline
 Proxy Contracts (Total) & 1,427,215  \\ \hline 
 Proxy Contracts (Filtered) & 13,088  \\ \hline 
 Regular Upgradeable Contracts & 7,470  \\ \hline
 UUPS & 403  \\ \hline
 Beacon & 352  \\ \hline
 \end{tabular}
 \caption{\label{tab:updata} Results of each \texttt{DELEGATECALL}-based upgrade pattern for the time-period {Sep-05-2020} to {Jul-20-2021} (2,064,595 blocks).}
 \vspace{-10pt}
 \end{table}
 
 Our measurements cover block number \texttt{10800000} to \texttt{12864595}, which corresponds to the time-period \texttt{Sep-05-2020} to \texttt{Jul-20-2021}, and are reported in Table~\ref{tab:updata}. While we found 1.4M unique proxy contracts, many of these share a common implementation contract and are part of the same larger upgradable system. As one example, the NFT marketplace OpenSea~\footnote{\url{https://opensea.io}} gives each user a unique proxy contract. After clustering contracts, we find 13K unique systems.  
 
 For the 8,225 upgradeable systems (regular, UUPS and beacon), we randomly sampled 150 contracts and manually verified they were upgradeable proxy contracts. We also sampled 150 contracts from the forwarders to verify they are not upgradeable, however we did find 2 false-negatives. Our model did not catch these contracts because a failure happened when decompiling them and our assignment checker detector in turn failed. Note that for UUPS contracts, the implementation contracts are much larger and harder to analyze than the proxy contract itself.


 %%%%%%%%%%%%%%%%%%%%%%%%
 %%%%%%%%%%%%%%%%%%%%%%%%

 \section{Finding the Admin}
\label{sec:governance}

If a contract is upgradeable, someone must be permissioned to conduct upgrades. We call this agent the admin of the contract. In the simplest case, the admin is a single Ethereum account controlled by a private signing key, called an externally owned account (EOA). A breach of this key could lead to malicious updates, as in the case of the lending and yield farming DeFi service Bent Finance ~\cite{bentFinanceHack}. Bent Finance deployed a \textit{Transparent Upgradeable Proxy} with an EOA admin that was breached (unconfirmed if via an external hack or insider attack). The EOA pushed an updated logic contract\footnote{\url{https://etherscan.io/address/0xb45d6c0897721bb6ffa9451c2c80f99b24b573b9}}  which moved tokens valued at \$12M USD into the attacker's account\footnote{0xd23cfffa066f81c7640e3f0dc8bb2958f7686d1f} and then upgraded the logic contract to a clean version to cover-up the attack. Based on \textit{The State of DeFi Security 2021}~\cite{certikReport} report by Certik,\footnote{\url{certik.com}} ``centralization risk'' is the most common attack vector for hacks of DeFi projects. 
 
Control over upgradeability typically falls into one of three categories: 

\begin{enumerate}
\item \textbf{Externally owned Address (EOA):}
One private key controls upgrades. It is highly centralized and one malicious admin or compromised private key could be catastrophic. It is also the fastest way to respond to incidents. An EOA may also pledge to delegate their actions to an off-chain consensus taken on any platform, such as verified users on \textit{Discord} or \textit{Snapshot}, however with no guarantee they will abide by it. In our measurements, we cannot distinguish this subtype as these are off-chain, social arrangements. 

\item \textbf{Multi-Signature Wallet:}
Admin privileges are assigned to a multi-signature wallet, requiring transactions signed by at least $m$ of a pre-specified $n$ EOAs.   This distributes trust, and tolerates some corruption of EOAs or loss of keys. There is no guarantee different EOAs are operated by different entities and may be security theatre put on by a single controlling entity.

\item \textbf{On-Chain Governance Voting:} A system issues a governance token and circulates it amongst its stakeholders. Updates are decided through a decentralized voting scheme where the weight of the vote from an EOA (or contract address) is proportionate to how many tokens it owns. This system is potentially highly decentralized, but the degree depends on the distribution of tokens (\eg if a single entity controls a majority of tokens, it is effectively centralized). Voting introduces friction: (1) a time delay to every decision---some critical functionality might bypass the vote and use quicker mechanisms (\eg global shutdown in MakerDAO), and (2) on-chain network fees for each vote cast.

\end{enumerate}

\subsection{Methodology}

We conduct our measurement on the 7,470 regular upgradeable contracts from Section~\ref{sec:proxyFinding}. The process can be divided into two main parts: finding the admin account's address and finding the admin type (EOA, multi-sig, or decentralized governance).

\paragraph{Finding the admin account's address.} EIP-1967 suggests specific arbitrary slots for upgradeable proxy contracts to store the \textit{admin address}.\footnote{Storage slot 0xb53127684a568b3173ae13b9f8a6016e243e63b6e8ee1178d6a717850b5d6103} We first check this specific storage slot using \texttt{eth\_getStorageAt} on the full node. If it is non-zero, we mark what is stored as the admin address. 
For non-EIP-1967 proxies, we use a process that is very similar to how we found the storage slot of the target address in Section~\ref{sec:proxyFinding}. We first find the function in which the admin can change the \textit{target address} (upgrade function). This function is critical and should only be called by the admin. We extract the access control check and mark the address authorized to run this function as the admin address.

\paragraph{Finding the admin type.} Having the admin address, we can check if the account is an EOA by invoking \texttt{eth\_getCode} on the address from the full node: if it is empty, it is an EOA. Otherwise, it is a contract address. The most common multisig contract is Gnosis Safe.\footnote{\url{https://gnosis-safe.io/}} We automatically mark the admin type as multi-sig if we detect Gnosis safe. We then switch the manual inspection to find other multi-signature wallets (\eg MultiSignatureWalletWithDailyLimit, \etc) and add them to the data set. 

In some cases, the admin address is itself a proxy contract---a pattern known as an Admin Proxy. This adds another layer of indirection. We are reusing our methodology for identifying proxy contracts to exact the real admin account, and the proceed as above. Further details of the methodology and implementation are provided below:

\subsubsection{EIP-1967.}
As mentioned above EIP-1967~\footnote{\url{https://eips.ethereum.org/EIPS/eip-1967}} suggested specific arbitrary slots for upgradeable proxy contracts to store implementation contract's address and \textit{Admin address}\footnote{Storage slot 0xb53127684a568b3173ae13b9f8a6016e243e63b6e8ee1178d6a717850b5d6103 for admin}.

In first step we use \textit{eth\_getStorageAt} method of an Ethereum full archival node to search the EIP-1967 specified storage slot for admins on our 7,470 proxy contracts. If the result of this method is non-zero it means that the proxy uses EIP-1967 standard because the specified storage slot is an arbitrary slot and one can store variable in this slot just by defining this slot which means that they used EIP-1967.

So, for non zero results, we capture the address which is the address of admin of the proxy. Now we try to find the type of these admin addresses. Having the address of the admin we use \textit{eth\_getCode} method to check the code of the admin account. If the code is empty, it means that this account is not a smart contract so it is an EOA. we find \textit{900} EOA admins that their proxy uses EIP-1967 standard.

The remained admin addresses are contract because their account keeps code. This contract can be multi signature smart contract wallets. The most widely used multi signature wallet is Gnosis Safe\footnote{\url{https://gnosis-safe.io/}} wallets. We automatically checked if the code of the admin address is the Gnosis wallet multi signature patterns. After picking Gnosis safe wallets we manually checked \%10 of the remained addresses to find if they used other patterns for their multi signature wallet and we found some other patterns (e.g. MultiSignatureWalletWithDailyLimit, etc.). After Finding all these types we checked the admin codes to see whether they are multi signature wallets. We find \textit{255} admin accounts that uses multi signature wallets as their admin.

There is another class of admin contracts named \textit{Admin Proxy} contracts. These admin proxy contracts are another layer of re-direction between the real admin and the Dapp's proxy contract. The admin proxy contracts are proxy contracts that redirect the messages from the real admin into the Dapp's proxy. The only person who can use admin proxy is the admin (a.k.a owner) of the admin proxy. So we first filter the admin proxy contracts using the codes we get from the previous part and then try to find the owner of the admin proxy contracts. The owner of admin proxy contract (the real admin) also can be EOA, Multi-sig or governance contract. Finding the owner of the admin proxy contract, we used \textit{eth\_getCode} method to check the code of these account and find out if they are EOAs or Multi-signatures or governance schemes. Doing this we find \textit{1202} EOA admin accounts and \textit{567} multi signature admins. We marked the remained proxy admin addresses as Governance/Not Known admin types and we have \textit{462} of them. There were also non admin proxy contracts which use EIP-1967 but they were not EOA or Multi signatures. We marked them as Governance/Not Known admin types and we have \textit{53} of them.

\subsubsection{Non EIP-1967.}
For proxy contracts which not use EIP-1967, the problem is we don't know where the admin address is saved in the proxy contract's storage (what is the storage slot of the admin address). It can be saved in a storage slot of the contract or be hardcoded in the smart contract\footnote{There are some other possible ways to store the admin address for instance saving it in another contract and each time make an external call to get the address but to our knowledge this pattern is not widely used as a standard}. 

So there are two ways that the admin address is saved in the proxy contract. It can be saved as a storage variables or it can be hardcoded as a fixed address.

In storage variable case, the first question is in which storage slot the admin address is stored. So, the first step is to find the storage slot of the admin address variable. Also for the fixed address we should find the fixed address of the admin directly.

To find the slot of the storage variable in which admin address is saved, we first find the function in which the proxy can be upgraded. For finding the upgrade function we exactly do what we did in \ref{sec:proxyFinding} part. We first find the storage variable in which we saved the implementation address and then we find a function in which the implementation address can be changed using the inputs of that specific function. 

The upgrade function of a proxy contract is a critical function and the only account that can call this function should be the admin of the proxy contract. So, there should be an access control check inside the upgrade function to check whether transaction sender is equal to the admin address or not.
So, after finding the upgrade function we search for conditionals that checks the caller of the transaction and by doing that we can find the admin address or the storage variable in which the admin address is stored.


If the admin address is stored in a storage variable, then we should find the storage slot of that specific storage variable. For finding the storage slot we do what we did in \ref{sec:proxyFinding} part by using \textit{def storage} function of the decompiler and check the storage slot of the storage variable we found, and the admin address is saved on it. 
Now we have the storage slot of the admin address and we should start doing all the things we did for EIP-1967 in the previous part. 
In the EIP-1967 the storage slot for admin address was pre-specified and we do not need to find the slot but in this case we use the above methodology to find the slot but the further steps are the same as EIP-1967.
So, by using the \textit{eth\_getCode} method for admin address inside the storage slot we find above, we can check wether the admin is EOA, Multi-sig, Governance, Proxy admin or not known.
In this part we find \textit{1313} EOA addresses and \textit{104} multi-sig admins. Also by checking proxy admins we find \textit{92} EOA addresses and \textit{16} Multi signatures that uses proxy admin as a level of indirection. 

In another case the admin address may be stored directly in a specific arbitrary storage slots. In this type the compiler will specify the address using the \textit{sha3} hash notation. In this case same as above we find the conditional check on the transaction sender and then find the storage slot in that line and hash of that pre-specified string. 
By finding this arbitrary storage slot and doing the same processes we did in the previous part we find \textit{2} EOA addresses and \textit{10} Multi-sig addresses.

The only case that is left is proxy contracts, in which the address of the admin is hardcoded inside them. It very straight forward. We find the upgrade function and the access control check on the caller of the transaction and then pick the fixed admin address and do the same processes mentioned above to find the admin types. There are \textit{49} EOAs, \textit{36} multi-signature admins and \textit{160} governance and not known admin addresses. 

\subsection{Results}

\begin{table}[t]
\centering
\begin{tabular}{|l|r|r|r|r|r|r|}
\hline      &\multicolumn{2}{c|}{EIP-1967} & \multicolumn{4}{c|}{Non-EIP-1967}         \\
\hline
Type 				& \makecell{Regular\\Admins} & \makecell{Admin\\Proxy} 	& \makecell{Regular\\Admins} 	& \makecell{Admin\\Proxy} & \makecell{Arbitrary\\Slots} &  \makecell{Fixed\\Address}  \\ \hline 
EOA 				& 900  	& 1202		& 1313		& 92			& 2		& 49 		\\ 
Multisig 			& 255  	& 567		& 104		& 16			& 10		& 36		\\ 
Governance/Other 	& 53  	& 462		& 			& 			& 		& 160 	\\ \hline
\end{tabular}
\caption{\label{tab:admindata} Results of each admin type in upgradeable contracts for the time-period {Sep-05-2020} to {Jul-20-2021} (2,064,595 blocks).}
\vspace{-10pt}
\end{table}

To sum up, of 7470 proxies, 3558 are controlled by an EOA address, 988 are controlled by a known multi-signature wallet, and 2924 addresses are remaining. Table~\ref{tab:admindata} breaks down each sub-category for these. Of the latter 2924 addresses, these are either decentralized governance or another unknown type. After manual inspection, we note some of the unknown contracts use undefined or new patterns for implementing multi-sig contracts; our model has false negatives in detecting multi-signatures. The results demonstrate significant centralization risk in upgradeability: 48\% of systems could be upgraded with the breach of a single signing key, and an additional 13\% by potentially a small number of signing keys.

%%%%%%%%%%
%%%%%%%%%%


\section{Attacking Universal Upgradeable Proxy Standard (UUPS) contracts} 
\label{sec:attackUUPS}
In this section, we described one of the use-cases of the dataset provided in section~\ref{sec:proxyFinding}. In Section~\ref{sysComplexity}, we discussed that one of the main challenges to the \texttt{DELEGATECALL}-based data separation pattern (Section~\ref{sec:delegatecall}) is that the constructor inside the implementation contract cannot initialize the proxy itself. So instead, there should be a regular function inside the implementation contract named \textit{Initialize} function that can be called just once after deployment by the proxy contract and has the same functionality as the constructor function. Therefore, the contract creator must call the initialize function quickly after deploying the proxy contract. The Initialize function does not have any access control because it is considered to be called once, and this function is responsible for defining the owner. So, before calling this function, the owner's address is not set, and there is no way to have an access control check for the sender. This is why there should be a check to ensure that this function can just be called once at deployment and not after. 
The proxy contract deployer will define and initialize the address of the contract owner via the initialize function. So if the deployer forgets to initialize the contract, any external address can call initialize function and change the owner of the contract to her desired address, and take control of the contract.

The \texttt{Initialize} function can also be called from the implementation contract itself (instead of calling the function by proxy contract). This call will alter the storage of the implementation contract and not the proxy contract. Suppose the deployer forgets to call the \texttt{Initialize} function directly from the implementation contract. Any malicious address can call this function from the implementation contract and change the owner inside the implementation contract, and take control of this contract. 
This malicious actor can change the storage of the implementation contract by calling functions inside it. However, it is not a risk to the system because the proxy contract is responsible for keeping the data in this system and not implementing it. There should be a risk to the system if this malicious actor can change the logic of the implementation contract or self-destruct it. Changing the logic of the implementation contract is not doable in typical cases because implementation contracts are not supposed to be upgradeable. However, there are ways to self-destruct the implementation contract. 

There are two main ways to self-destruct a contract: 1)if the implementation contract has \texttt{SELFDESTRUCT} inside its logic and by calling it. 2)Having a \texttt{DELEGATECALL} or \texttt{CALLCODE} to another contract that has \texttt{SELFDESTRUCT} logic inside~\cite{frowisnot}.

So, we should check if the implementation contract uses \texttt{SELFDESTRUCT} or has \texttt{DELEGATECALL} or \texttt{CALLCODE} to an address that a malicious party can control. If there is a way, the malicious party can self-destruct the implementation contract, and all calls to the proxy will fail. It is a Denial of Service (DoS) attack on the Dapp.
If the Dapp has an upgrade function inside its proxy contract, then the admin of the proxy contract can upgrade into a new version of the implementation contract. This attack was explained in December 2020 by  Trail of Bits team when they audited the code of Aave, a lending project~\cite{aaveBreak}.

Nevertheless, what if the upgrade function is inside the implementation contract and not the proxy contract. As mentioned in Section~\ref{sec:delegatecall}, in UUPS upgradeable contracts, the upgrade function resides in the implementation contract. So there is no way to upgrade the system by the proxy contract. Therefore, if an attacker takes control of the implementation contract by calling the initialize function directly from the implementation contract and then self-destruct it, there is no way to upgrade it. Consequently, the proxy will be locked forever. 
All UUPS contracts that used the Openzepplin UUPS library, whose implementation contact is not initialized, are susceptible to this attack. Because there is a function in the implementation contract of this library named \textit{upgradeToAndCall}, in which the owner can change a target address and then delegate call into the newly changed target address. This attack vector was found in September 2021 and announced by OpenZepplin team~\cite{securityAdvise,uupsAttacks}. There is an easy way to mitigate this attack by calling the initialize function directly from the implementation contract. 

We try to check all UUPS contracts that we find in Section~\ref{sec:proxyFinding} that if any of them can be exploited in this way. We check all of them manually, and the method of checking them is described below:

\begin{enumerate}
\item Check if the implementation contract is not initialized
\item Find initialize function inside the implementation contract
\item check if anybody can call this initialize function directly from the implementation contract and change the owner of the contract
\begin{itemize}
\item Filter those that have a modifier that blocks direct calls to the implementation contract (there is a modifier that just let transactions that come from the proxy contract and blocks direct calls to the implementation contract itself)
\end{itemize}
\item Check if there is a way inside the implementation to self-destruct 
\item Check if there is a function in the implementation contract which has a delegate call to a target address
\item Check if the target address is changeable by a malicious actor
\end{enumerate}

After reviewing the list above, we found 15 contracts in our data set that were exploitable until September 9, 2021. The openzeppelin team patched them by initializing the contract. An attacker could deploy a new malicious contract that executes self-destruct on any calls to it. Then take control of the implementation contract by calling the initialize function them. Afterward, the attacker finds the function inside the implementation contract with a delegate call inside it and finds the target address. There should be a function inside the implementation contract to change the address to the malicious contract that the attacker deployed recently. The attacker calls the function to execute a delegate call into the malicious contract and then self-destruct the implementation contract.

We find 61 UUPS contracts that are not initialized, and anybody can take control of these implementation contracts. However, because these contracts do not use delegate calls or self-destruct, they are not exploitable by this type of attack.

%%%%%%%%%%%%%%%%%%%%%%
%%%%%%%%%%%%%%%%%%%%%%

\section{Concluding Remarks}

In this chapter, we find that \texttt{DELEGATECALL}-based data separation is the most prominent upgrade pattern in Ethereum in recent years. Our evaluation framework gives some hint as to why this is the case. It avoids the need for a social upgrade, as in contract migration or the \texttt{CALL}-based pattern (without a proxy). \texttt{CREATE2}-based metamorphosis was recently made possible (with the introduction of \texttt{CREATE2}) and its use might grow over time, however it shares one major drawback with contract migration: the need to migrate the whole state from the old contract for each update, even if the update is makes minor changes to the logic of the contract. Metamorphic contracts also run the risk of Ethereum removing the \texttt{SELFDESTRUCT} opcode they rely on.  A drawback of \texttt{CALL}-based patterns is the heavy instrumentation each new contract needs before it can be deployed, whereas in a \texttt{DELEGATECALL}-based (along with migration and \texttt{CREATE2}-based) upgrade pattern, developers can simply deploy the new logic contract exactly as it is written. Putting these reasons together, \texttt{DELEGATECALL}-based pattern is an attractive option on balance. 


The main take-away from studying upgradeability on Ethereum is that immutability, as a core property of blockchain, is oversold. Immutability has already been criticized for being dependent on consensus---both technical and social~\cite{walch2016path}---however the widespread use of upgradeability patterns further degrades immutability. Finally, as we show, the prominence of contracts that can be upgraded with a single private key (\ie externally owned account) calls into question how decentralized our DApps (decentralized applications) really are. If the upgrade process is corrupted through a key theft or by a rogue insider, the whole logic of the contract can be changed to the attacker's benefit. 

One recent application of our research was finding all contracts that implement the UUPS upgrade pattern, which become important when a vulnerability was discovered in one of the best-known libraries for implementing UUPS. We describe how we can find potentially vulnerable contracts in Section~\ref{sec:attackUUPS}. While others had found some contracts by looking for specific artifacts left by the UUPS library, we improved the state of the art by looking for the generic pattern of UUPS. 

A final discussion point concerns Layer 2 (L2) solutions, such as optimistic rollups and zk-rollups~\cite{mccorry2021sok}. For the readers that are already familiar with them, their central component is a bridge contract that let computations be performed off of Ethereum (layer 1) and have just the outputs validated on Ethereum. If the bridge contracts is upgradeable, the rules for accepting L2 state are also upgradeable which means every L2 contract is de facto upgradeable even it does not implement an upgrade pattern. We saw Ethereum override the consensus of the network to revert the DAO hack, which was a rare and contentious event. If a similar attack happened on a L2, reverting would be much simpler and not require a hard fork: the L2 could simply update the bridge contract. For this reason, the consensus override upgrade pattern may be less rare in the future. 


\include{chapters/chapter4/chapter4}
\include{chapters/chapter5/chapter5}


%%%%%%%%%%%%%%%%%%%%%%%%
\chapter{Conclusion and Future Works}
\label{chap:conclusion}

In this dissertation, we present an analysis of upgradeability, oracles, and stablecoins in the Ethereum blockchain, which are three components in Ethereum that add centrality by putting a trust point in the system and increasing the attack vectors of the decentralized applications ecosystem.

We describe six different patterns to add upgradeability features to immutable smart contracts and evaluate each method, to give the smart contract developers an insight into the pros and cons of each method. We prepared a dataset of upgradeable proxy contracts-- the most widely used upgrade pattern in Ethereum at the time. We also investigate the admin type of a dataset of upgradeable contracts. The main takeaway from studying upgradeability on Ethereum is that immutability, as a core property of blockchain, is oversold. Also, we show that almost 64\% of smart contracts are in the control of a centralized agent who can decide to change the whole logic of the system.

We also describe a specialized modular framework to analyze the oracle system in Ethereum. After our systematization, we claim that diversity in software promotes resilience in the system. Most Ethereum projects depend on the data provided by Chainlink oracle, which can be a single point of failure for the Ethereum ecosystem. There is a need to diversify the oracle systems. Also, we propose that to have a secure oracle, it should have a mechanism to ensure that the profit from corruption is lesser than the cost of corruption. Nevertheless, the problem is that one can never capture the full extent of the potential profit because attackers may profit outside of Ethereum by attacking oracles on Ethereum. 

Researching the stable coins shows how price changes on the underlying assets (\eg ETH) may negatively affect the stablecoin systems. A myriad of users who uses these stablecoins assumes that the price of stablecoins is stable. However, they do not know about the under the hood of the stablecoin systems and the risks regarding the price of the stablecoins. We also show a different design spectrum other than a liquidation to ensure that the stablecoin is pegged to a US dollar.

For future work, the dataset provided in our research can be used to show undiscovered facts about upgrade events in Ethereum. For instance, one can research the number of upgrades of the Dapps in Ethereum and investigate the reason behind upgrades. It may show a new way to find hacks or incidents that the developer team tried to hide. In oracle research, many theoretical vulnerabilities are proposed to different modules of oracles which one can research to show the effectiveness of each of the vulnerabilities. In stablecoin research, it would be helpful to have research results on the most suitable financial model for the ETH/USD price rate (\eg drift-diffusion or GARCH) for us to use. Future work could also examine the benefits of building a \dai alternative based on red-black coins but using different design parameters.

%%%%%%%%%%%%%%%%%%%%%%%%%%%%%%%%%%%%%%%%%%%%%%%%
%% Bibliography
%%%%%%%%%%%%%%%%%%%%%%%%%%%%%%%%%%%%%%%%%%%%%%%%
\clearpage
\phantomsection
\addcontentsline{toc}{chapter}{Bibliography}  %  Add Bibliography to TOC
\singlespacing % save space in the bibliography
\bibliographystyle{abbrv}
\bibliography{references,bib/pulp.bib,bib/new.bib,bib/bib.bib}



%%%%%%%%%% Appendices %%%%%%%%%%%%%%%%
% ---- Appendix settings. Please Do NOT change them. -----
\appendix
\setcounter{table}{0}		% reset the table counter
\setcounter{figure}{0}		% reset the figure counter
\renewcommand{\thefigure}{\Alph{chapter}.\arabic{figure}} 	% numbering the a figure in Appendix as Figure A.2, Figure B.1, etc.
\renewcommand{\thetable}{\Alph{chapter}.\arabic{table}}		% numbering the a table in Appendix as Table A.2, Table B.1, etc.

%%%%%%%%%% Body of Appendix %%%%%%%%%%%%%%%%
%\begin{appendices}
%\doublespacing

%\chapter{First Appendix}
%\label{chap:apdx1}



%\chapter{Concordia Logos}
%\label{chap:logos}
%\begin{figure}[h!]
%	\centering
%	\includegraphics{logos/Concordia_University_logo}
%	\caption{Concordia University}
%\end{figure}
%\vspace{2em}
%\begin{figure}[h!]
%	\centering
%	\includegraphics{logos/Concordia_GinaCody_vertical}
%	\caption{Gina Cody School of Engineering and Computer Science (vertical)}
%\end{figure}
%\vspace{2em}
%\begin{figure}[h!]
%	\centering
%	\includegraphics{logos/Concordia_GinaCody_horizontal}
%	\caption{Gina Cody School of Engineering and Computer Science (horizontal)}
%\end{figure}

%\end{appendices}

\end{document}